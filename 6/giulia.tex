\documentclass[]{article}
\usepackage[utf8]{inputenc}
\usepackage[english]{babel}

\usepackage[]{csvsimple}

\usepackage{ragged2e}
\usepackage[left=25mm, right=25mm, top=15mm]{geometry}
\geometry{a4paper}
\usepackage{graphicx}
\usepackage{booktabs}
\usepackage{paralist}
\usepackage{subfig} 
\usepackage{fancyhdr}
\usepackage{amsmath}
\usepackage{amssymb}
\usepackage{amsfonts}
\usepackage{amsthm}
\usepackage{mathtools}
\usepackage{enumitem}
\usepackage{titlesec}
\usepackage{braket}
\usepackage{gensymb}
\usepackage{url}
\usepackage{hyperref}
\usepackage{csquotes}
\usepackage{multicol}
\usepackage{graphicx}
\usepackage{wrapfig}
\usepackage{babel}
\usepackage{caption}
\captionsetup{font=small}
\pagestyle{fancy}
\renewcommand{\headrulewidth}{0pt}
\lhead{}\chead{}\rhead{}
\lfoot{}\cfoot{\thepage}\rfoot{}
\usepackage{sectsty}
\usepackage[nottoc,notlof,notlot]{tocbibind}
\usepackage[titles,subfigure]{tocloft}
\renewcommand{\cftsecfont}{\rmfamily\mdseries\upshape}
\renewcommand{\cftsecpagefont}{\rmfamily\mdseries\upshape}

\let\oldsection\section% Store \section
\renewcommand{\section}{% Update \section
	\renewcommand{\theequation}{\thesection.\arabic{equation}}% Update equation number
	\oldsection}% Regular \section
\let\oldsubsection\subsection% Store \subsection
\renewcommand{\subsection}{% Update \subsection
	\renewcommand{\theequation}{\thesubsection.\arabic{equation}}% Update equation number
	\oldsubsection}% Regular \subsection

\newcommand{\abs}[1]{\left\lvert#1\right\rvert}
\newcommand{\norm}[1]{\left\lVert#1\right\rVert}

\newcommand{\g}{\text{g}}
\newcommand{\m}{\text{m}}
\newcommand{\cm}{\text{cm}}
\newcommand{\mm}{\text{mm}}
\newcommand{\s}{\text{s}}
\newcommand{\N}{\text{N}}
\newcommand{\Hz}{\text{Hz}}

\newcommand{\virgolette}[1]{``\text{#1}"}
\newcommand{\tildetext}{\raise.17ex\hbox{$\scriptstyle\mathtt{\sim}$}}


\renewcommand{\arraystretch}{1.2}

\addto\captionsenglish{\renewcommand{\figurename}{Fig.}}
\addto\captionsenglish{\renewcommand{\tablename}{Tab.}}

\DeclareCaptionLabelFormat{andtable}{#1~#2  \&  \tablename~\thetable}

\title{%
    \Huge Misura dell'indice di rifrazione di un vetro con lo spettrometro a prisma \\
    \Large Laboratorio di Ottica, Elettronica e Fisica Moderna \\ C.d.L. in Fisica, a.a. 2023-2024 \\ Università degli Studi di Milano}
\author{\LARGE Lucrezia Bioni, Leonardo Cerasi, Giulia Federica Bianca Coppi \\ Matricole: 13655A, 11410A, 11823A}
\date{23 novembre 2023}

\begin{document}

    \maketitle

    \section{Introduzione}

    \subsection{Scopo}

    Mediante l'utilizzo di un prisma a sezione isoscele, si vuole misurare l'indice di rifrazione del materiale che lo compone. Si vuole inoltre verificare la legge di dispersione secondo la formula di Cauchy:
    \begin{equation}
        \label{cauchy}
        n(\lambda) = A + \frac{B}{\lambda^2}
    \end{equation}
    Dove dove $n$ è l'indice di rifrazione, $\lambda$ è la lunghezza d'onda, $A$ e $B$ sono i coefficienti che possono essere determinati per un materiale interpolando l'equazione ad indici di rifrazione misurati per lunghezze d'onda note.

    \subsection{Metodo}

    In seguito alla misurazione dello spettro di emissione della lampada ai vapori di mercurio - effettuata con il reticolo di diffrazione -, si utilizzano le lunghezze d'onda trovate per misurare l'indice di rifrazione del materiale vetroso che compone il prisma. \\
    Tale misurazione viene effettuata attraverso il metodo della deviazione minima: si può ricavare la dipendenza dell'angolo $\delta$ in funzione dell'angolo di incidenza $i$, dimostrando inoltre che la funzione $\delta (i)$ presenta un minimo. La condizione di deviazione minima si presenta nel momento in cui viene soddisfatta l'equazione:

    \begin{equation}
        \label{dev_minima}
        \cos i \cdot \cos r' = \cos r \cdot \cos i'
    \end{equation}
    Dove $i$ è l'algolo di incidenza, $i'$ è l'angolo di emergenza $r$ è l'angolo di rifrazione sulla faccia di entrata del prisma e $r'$ l'angolo di incidenza sulla seconda faccia del prisma.

    Queste quantità sono legate a $\delta$ dalle seguenti relazioni:

    \begin{equation}
        \label{relazioni_delta}
        r + r' = \alpha
        \delta = i + i' - \alpha
    \end{equation}
    Dove $\alpha$ è l'angolo al vertice del prisma.

    L'indice di rifrazione del prisma, in condizioni di minima deviazione, risulterà essere quindi:

    \begin{equation}
        \label{indice_rifrazione}
        n(\lambda)=\frac{\sin \frac{\alpha + \delta _m} {2}}{\sin \frac{\alpha}{2}}
    \end{equation}
    Dove $n(\lambda)$ è l'indice di rifrazione del materiale in funzione della lunghezza d'onda $\lambda$ considerata, $\alpha$ l'angolo al vertice della sezione del prisma, $\delta _m $ l'angolo di minima deviazione della lunghezza d'onda considerata.


    \section{Analisi dati}
    \subsection{Elaborazione dati}
    \subsubsection{Valore dell'angolo $\alpha$ del prisma}
    Dalla misura della posizione del fascio di luce riflessa da due delle facce del prisma, si ricava la posizione dell'angolo $\alpha$ compreso tra le due facce attraverso la seguente relazione:
    \begin{equation}
        \label{alpha}
        \alpha = 180° - \Delta\theta
    \end{equation}
    Dove $\Delta\theta=\theta_2-\theta_1$, e $\theta_1$ è la posizione angolare del fascio riflesso dalla prima faccia, mentre $\theta_2$ è la posizione angolare del fascio riflesso dalla seconda.
    Tale calcolo è stato eseguito per ogni set di misure di $\theta_1$ e $\theta_2$, ed è stata effettuata una media aritmetica per determinare il valore finale di $\alpha$, pari a: 
    \begin{equation}
        \label{alpha-value}
        \alpha = 59 \degree \, 53' \pm 12'
    \end{equation}
    Dove l'errore è stato attribuito come da Par. \ref{par:alpha_err}.

    \subsection{Stima degli errori}
    \subsubsection{Valore dell'angolo $\alpha$ del prisma}
    L'errore attribuito ai singoli valori di $\alpha$ è stato ottenuto propagando l'errore su $\theta_1$ e $\theta_2$ nella \ref{alpha}:
    \begin{equation}
        \label{alpha-err}
        \alpha = \sqrt{2} \cdot \sigma_{\theta} 
    \end{equation}
    Ai singoli valori di $\alpha$ ottenuti per ciascun set di misure è stata attribuita come incertezza 
     la risoluzione dello strumento, pari a $1'$. Al valore finale di $\alpha$
    Ad $\alpha$ è stato attribuito come errore il massimo tra la deviazione standard della singola misura e la risoluzione dello strumento, pari a $1'$.
    \label{par:alpha_err}

\end{document}