\documentclass[]{article}
\usepackage[utf8]{inputenc}
\usepackage[english]{babel}

\usepackage[]{csvsimple}
\usepackage[]{float}

\usepackage{ragged2e}
\usepackage[left=25mm, right=25mm, top=15mm]{geometry}
\geometry{a4paper}
\usepackage{graphicx}
\usepackage{booktabs}
\usepackage{paralist}
\usepackage{subfig} 
\usepackage{fancyhdr}
\usepackage{amsmath}
\usepackage{amssymb}
\usepackage{amsfonts}
\usepackage{amsthm}
\usepackage{mathtools}
\usepackage{enumitem}
\usepackage{titlesec}
\usepackage{braket}
\usepackage{gensymb}
\usepackage{url}
\usepackage{hyperref}
\usepackage{csquotes}
\usepackage{multicol}
\usepackage{graphicx}
\usepackage{wrapfig}
\usepackage{babel}
\usepackage{caption}
\captionsetup{font=small}
\pagestyle{fancy}
\renewcommand{\headrulewidth}{0pt}
\lhead{}\chead{}\rhead{}
\lfoot{}\cfoot{\thepage}\rfoot{}
\usepackage{sectsty}
\usepackage[nottoc,notlof,notlot]{tocbibind}
\usepackage[titles,subfigure]{tocloft}
\renewcommand{\cftsecfont}{\rmfamily\mdseries\upshape}
\renewcommand{\cftsecpagefont}{\rmfamily\mdseries\upshape}

\let\oldsection\section% Store \section
\renewcommand{\section}{% Update \section
	\renewcommand{\theequation}{\thesection.\arabic{equation}}% Update equation number
	\oldsection}% Regular \section
\let\oldsubsection\subsection% Store \subsection
\renewcommand{\subsection}{% Update \subsection
	\renewcommand{\theequation}{\thesubsection.\arabic{equation}}% Update equation number
	\oldsubsection}% Regular \subsection

\newcommand{\abs}[1]{\left\lvert#1\right\rvert}
\newcommand{\norm}[1]{\left\lVert#1\right\rVert}

\newcommand{\g}{\text{g}}
\newcommand{\m}{\text{m}}
\newcommand{\cm}{\text{cm}}
\newcommand{\mm}{\text{mm}}
\newcommand{\s}{\text{s}}
\newcommand{\N}{\text{N}}
\newcommand{\Hz}{\text{Hz}}

\newcommand{\virgolette}[1]{``\text{#1}"}
\newcommand{\tildetext}{\raise.17ex\hbox{$\scriptstyle\mathtt{\sim}$}}


\renewcommand{\arraystretch}{1.2}

\addto\captionsenglish{\renewcommand{\figurename}{Fig.}}
\addto\captionsenglish{\renewcommand{\tablename}{Tab.}}

\DeclareCaptionLabelFormat{andtable}{#1~#2  \&  \tablename~\thetable}

\title{%
    \Huge Misura dell'indice di rifrazione di un vetro con lo spettrometro a prisma \\
    \Large Laboratorio di Ottica, Elettronica e Fisica Moderna \\ C.d.L. in Fisica, a.a. 2023-2024 \\ Università degli Studi di Milano}
\author{\LARGE Lucrezia Bioni, Leonardo Cerasi, Giulia Federica Bianca Coppi \\ Matricole: 13655A, 11410A, 11823A}
\date{23 novembre 2023}

\begin{document}

    \maketitle

    \section{Introduzione}

    \subsection{Scopo}

    Mediante l'utilizzo di un prisma a sezione isoscele, si vuole misurare l'indice di rifrazione del materiale che lo compone. Si vuole inoltre verificare la legge di dispersione secondo la formula di Cauchy:
    \begin{equation}
        \label{cauchy}
        n^2(\lambda) = A + \frac{B}{\lambda^2}
    \end{equation}
    Dove dove $n$ è l'indice di rifrazione, $\lambda$ è la lunghezza d'onda, $A$ e $B$ sono i coefficienti che possono essere determinati per un materiale interpolando l'equazione ad indici di rifrazione misurati per lunghezze d'onda note.

    \subsection{Metodo}

    In seguito alla misurazione dello spettro di emissione della lampada ai vapori di mercurio - effettuata con il reticolo di diffrazione -, si utilizzano le lunghezze d'onda trovate per misurare l'indice di rifrazione del materiale vetroso che compone il prisma. \\
    Tale misurazione viene effettuata attraverso il metodo della deviazione minima: si può ricavare la dipendenza dell'angolo $\delta$ in funzione dell'angolo di incidenza $i$, dimostrando inoltre che la funzione $\delta (i)$ presenta un minimo. La condizione di deviazione minima si presenta nel momento in cui viene soddisfatta l'equazione:

    \begin{equation}
        \label{dev_minima}
        \cos i \cdot \cos r' = \cos r \cdot \cos i'
    \end{equation}
    Dove $i$ è l'algolo di incidenza, $i'$ è l'angolo di emergenza $r$ è l'angolo di rifrazione sulla faccia di entrata del prisma e $r'$ l'angolo di incidenza sulla seconda faccia del prisma.

    Queste quantità sono legate a $\delta$ dalle seguenti relazioni:

    \begin{equation}
        \label{relazioni_delta}
        r + r' = \alpha
        \delta = i + i' - \alpha
    \end{equation}
    Dove $\alpha$ è l'angolo al vertice del prisma.

    L'indice di rifrazione del prisma, in condizioni di minima deviazione, risulterà essere quindi:

    \begin{equation}
        \label{indice_rifrazione}
        n(\lambda)=\frac{\sin \frac{\alpha + \delta _m} {2}}{\sin \frac{\alpha}{2}}
    \end{equation}
    Dove $n(\lambda)$ è l'indice di rifrazione del materiale in funzione della lunghezza d'onda $\lambda$ considerata, $\alpha$ l'angolo al vertice della sezione del prisma, $\delta _m $ l'angolo di minima deviazione della lunghezza d'onda considerata.


    \section{Analisi dati}
    \subsection{Elaborazione dati}
    \subsubsection{Angolo $\alpha$ del prisma}
    Dalla misura della posizione del fascio di luce riflessa da due delle facce del prisma, si ricava la posizione dell'angolo $\alpha$ compreso tra le due facce attraverso la seguente relazione:
    \begin{equation}
        \label{alpha}
        \alpha = 180° - \Delta\theta
    \end{equation}
    Dove $\Delta\theta=\theta_2-\theta_1$, e $\theta_1$ è la posizione angolare del fascio riflesso dalla prima faccia, mentre $\theta_2$ è la posizione angolare del fascio riflesso dalla seconda.
    Tale calcolo è stato eseguito per ogni set di misure di $\theta_1$ e $\theta_2$, ed è stata effettuata una media aritmetica per determinare il valore finale di $\alpha$, pari a: 
    \begin{equation}
        \label{alpha-value}
        \alpha = 59 \degree \, 53' \pm 12'
    \end{equation}
    Dove l'errore è stato attribuito come da Par. \ref{par:alpha_err}.

    \subsubsection{Angolo di deviazione minima}
    Per determinare la posizione angolare $\theta_0$ del cannocchiale nella direzione da cui proviene l'immagine diretta della fenditura, si è eseguita la media aritmetica tra i valori di $\theta_0$ misurati (\ref{}):
    \begin{equation}
        \label{theta-0}
        \theta_0 = -(1 \degree \, 18' \, 0'' \pm 40'') 
    \end{equation}
    Per ciascuna lunghezza d'onda dello spettro del mercurio, si determina l'angolo di inversione $\delta$ del moto dell'immagine osservata mediante il cannocchiale attraverso la seguente relazione:
    \begin{equation}
        \label{delta}
        \delta = | \theta_0 - \theta_{\lambda}|
    \end{equation}
    Dove $\theta_{\lambda}$ è la posizione angolare misurata del punto di inversione del moto.
    Attraverso la media aritmetica dei valori di $\delta$ ottenuti, se ne determina la miglior stima. I valori ottenuti di $\delta$ per ciascuna lunghezza d'onda osservata, con le loro incertezze (ricavate come da Par. \ref{par:dev_min_err}), sono riportati nella seguente tabella:
    \begin{table} [H]
        \centering
        \begin{tabular}{||c|c||}
            \hline
            Colore & $\delta \pm \sigma_{\delta}$\\
            \hline \hline
            Viola 1 & $ 74 \degree \, 9' \, 0'' \pm 1' \, 8'' $ \\\hline
            Viola 2 & $ 73 \degree \, 53'\, 0'' \pm 1' \, 44'' $ \\\hline
            Indaco & $ 71 \degree \, 47'\, 0'' \pm 27'  $ \\\hline
            Ciano & $ 68 \degree \, 58'\, 0'' \pm 1' \, 32'' $ \\\hline
            Verde & $ 67 \degree \, 16'\, 0'' \pm 23'' $ \\\hline
            Giallo 1 & $ 66 \degree \, 33'\, 0'' \pm 23'' $ \\\hline
            Giallo 2 & $ 66 \degree \, 32' \, 0''\pm 23'' $ \\\hline
        \end{tabular}
        \caption{Valori di $\delta$ e relativi errori.}
        \label{d-values}
    \end{table}

    \subsubsection{Indice di rifrazione del vetro}
    Ottenuti i valori dell'angolo di deviazione minima $\delta$ per ciascuna lunghezza d'onda e dell'angolo $\alpha$ al vertice del prisma, attraverso la relazione \ref{indice_rifrazione}, si ricavano i seguenti valori di indice di rifrazione del vetro del prisma $n$ in funzione della lunghezza d'onda $\lambda$:
    \begin{table} [H]
        \centering
        \begin{tabular}{||c|c|c||}
            \hline
            Colore & $\lambda \pm \sigma_{\lambda} [\cdot 10^{-9} \text{m}]$ & $ n(\lambda) \pm \sigma_n $\\
            \hline \hline
            Viola 1  & $404.32 \pm 0.08 $ & $1.845 \pm 0.004$   \\\hline
            Viola 2  & $407.70 \pm 0.10 $ & $1.843 \pm 0.004$    \\\hline
            Indaco   & $435.57 \pm 0.08 $ & $1.828 \pm 0.004$  \\\hline
            Ciano    & $491.21 \pm 0.08 $ & $1.807 \pm 0.004$  \\\hline
            Verde    & $545.44 \pm 0.08 $ & $1.794 \pm 0.004$ \\\hline
            Giallo 1 & $576.46 \pm 0.08 $ & $1.789 \pm 0.004$   \\\hline
            Giallo 2 & $578.41 \pm 0.08 $ & $1.789 \pm 0.004$   \\\hline
        \end{tabular}
        \caption{Valori di $n(\lambda)$.}
        \label{n-values}
    \end{table}
    Per verificare la relazione di Cauchy \ref{cauchy}, sono stati riportati sul grafico $riferimento-al-grafico$ i valori ottenuti dalle misure e dalla loro elaborazione. In particolare, si è posto sulle ascisse il termine $\frac{1}{\lambda^2}$ e sulle ordinate il valore $n^2$. Attraverso la regressione lineare pesata si sono ottenuti come valori del coefficiente angolare $A$ e del termine noto $B$ i seguenti:
    \begin{equation}
        \label{A}
        A = (3.002 \pm 0.006) 
    \end{equation}
    \begin{equation}
        \label{B}
        B = (6.5 \pm 0.1) \cdot 10^{-14} \text{m}^2
    \end{equation}
    Per verificare l'effettivo andamento lineare dei risultati ottenuti è stato effettuato un test del $\chi^2$:
    \begin{equation}
        \label{chi2}
        \chi^2 = 5.623 \cdot 10^{-1} 
    \end{equation}
    Tale valore restituisce una compatibilità con un andamento lineare di probabilità $76.07 \% $.
    
    \subsection{Stima degli errori}
    \subsubsection{Angolo $\alpha$ del prisma}
    L'errore attribuito ai singoli valori di $\alpha$ è stato ottenuto propagando l'errore su $\theta_1$ e $\theta_2$ nella \ref{alpha}:
    \begin{equation}
        \label{alpha-err}
        \alpha = \sqrt{2} \cdot \sigma_{\theta} 
    \end{equation}
    Al valore finale di $\alpha$ è stata attribuita come incertezza la deviazione standard della media delle misure effettuate. Si è scelto di attribuire l'incertezza statistica come errore poiché superiore all'incertezza sistematica, pari a
    \begin{equation}
        \label{alpha-err-sist}
        \sigma_{\alpha, sist}= \frac{\sqrt{2} \cdot \sigma_{\theta}}{\sqrt{10}}=27''
    \end{equation}
    \label{par:alpha_err}

    \subsubsection{Angolo di deviazione minima}
    L'errore attribuito al valore medio di $\theta_0$ è stato ricavato attraverso la deviazione standard della media delle misure effettuate. Tale valore è superiore all'incertezza sistematica, ottenuta dalla propagazione dell'errore sulla singola misura di $\theta_0$ nella formula per calcolare la media aritmetica:
    \begin{equation}
        \label{theta0-err-sist}
        \sigma_{\theta_0,sist} = \frac{\sigma_{\theta_0}}{\sqrt{10}} = 19''
    \end{equation}
    Dove $\sigma_{\theta_0}$ è l'incertezza attribuita alle singole misure di $\theta_0$.
    L'errore sul valore di $\delta$ ottenuto per i primi quattro colori osservati (Viola 1, Viola 2, Indaco e Ciano) è stato attribuito attraverso la deviazione standard della medie delle misure effettuate. Tale valore è risultato superiore rispetto all'incertezza sistematica, calcolabile attraverso la propagazione degli errori su $\theta_0$ e $\theta_{\lambda}$ nella \ref{delta}:
    \begin{equation}
        \label{delta-err-sist}
        \sigma_{\delta, sist}= \frac{\sqrt{ \sigma_{\theta_0}^2 + \sigma_{\theta_{\lambda}}^2 }}{\sqrt{10}} = 23''
    \end{equation}
    Dove $\sigma_{\theta_0}$ è l'incertezza attribuita alle singole misure di $\theta_0$ e $\sigma_{\theta_{\lambda}}$ è l'incertezza attribuita alle singole misure di $\theta_{\lambda}$.
    Nel caso degli ultimi 3 colori (Verde, Giallo 1 e Giallo 2), l'incertezza sistematica risulta invece superiore a quella statistica, ed è dunque stata attribuita come errore su $\delta$. Le incertezze statistiche sono riportate nella seguente tabella:
    \begin{table} [H]
        \centering
        \begin{tabular}{||c|c||}
            \hline
            Colore & $\sigma_{\delta_{stat}}$\\
            \hline \hline
            Verde & $ 0.3'' $ \\\hline
            Giallo 1 & $ 0.2'' $ \\\hline
            Giallo 2 & $ 0.2''  $ \\\hline
        \end{tabular}
        \caption{Valori dell'incertezza statistica su $\delta$.}
        \label{d-err-stat}
    \end{table}
    \label{par:dev_min_err}

    \subsubsection{Indice di rifrazione del vetro}
    L'incertezza attribuita a ciascun valore dell'indice di rifrazione del vetro $n$ è stata ottenuta mediante propagazione degli errori su $\delta$ e $\alpha$ nella \ref{indice_rifrazione}:
    \begin{equation}
        \label{sigma-n}
        \sigma_n = \sqrt{ \left(\frac{\sin{\frac{\delta}{2}}}{\cos{\alpha}-1}\right)^2 \cdot \sigma_{\alpha}^2 + \left(\frac{cos{\frac{\delta + \alpha}{2}}}{\sin{\frac{\alpha}{2}}}\right)^2 \cdot \sigma_{\delta}^2   }
    \end{equation}
    Dove $\sigma_{delta}$ è l'incertezza attribuita all'angolo di deviazione minima $\delta$ e $\sigma_{\alpha}$ è l'incertezza attribuita all'angolo al centro del prisma $\alpha$.

    \section{Conclusioni}
    I risultati ottenuti hanno permesso di verificare con ottima compatibilità la relazione di Cauchy.

\end{document}