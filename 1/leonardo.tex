\documentclass[]{article}
\usepackage[utf8]{inputenc}
\usepackage[english]{babel}

% imports %%%%%%%%%%%%%%%%%%%%%%%%%%%%%%%%%%%%%%%%%%%%%%%%%%%%%%

\usepackage[]{csvsimple}

\usepackage{ragged2e}
\usepackage[left=25mm, right=25mm, top=15mm]{geometry}
\geometry{a4paper}
\usepackage{graphicx}
\usepackage{booktabs}
\usepackage{paralist}
\usepackage{subfig} 
\usepackage{fancyhdr}
\usepackage{amsmath}
\usepackage{amssymb}
\usepackage{amsfonts}
\usepackage{amsthm}
\usepackage{mathtools}
\usepackage{enumitem}
\usepackage{titlesec}
\usepackage{braket}
\usepackage{gensymb}
\usepackage{url}
\usepackage{hyperref}
\usepackage{csquotes}
\usepackage{multicol}
\usepackage{graphicx}
\usepackage{wrapfig}
\usepackage{babel}
\usepackage{caption}
\captionsetup{font=small}
\pagestyle{fancy}
\renewcommand{\headrulewidth}{0pt}
\lhead{}\chead{}\rhead{}
\lfoot{}\cfoot{\thepage}\rfoot{}
\usepackage{sectsty}
\usepackage[nottoc,notlof,notlot]{tocbibind}
\usepackage[titles,subfigure]{tocloft}
\renewcommand{\cftsecfont}{\rmfamily\mdseries\upshape}
\renewcommand{\cftsecpagefont}{\rmfamily\mdseries\upshape}

\let\oldsection\section% Store \section
\renewcommand{\section}{% Update \section
	\renewcommand{\theequation}{\thesection.\arabic{equation}}% Update equation number
	\oldsection}% Regular \section
\let\oldsubsection\subsection% Store \subsection
\renewcommand{\subsection}{% Update \subsection
	\renewcommand{\theequation}{\thesubsection.\arabic{equation}}% Update equation number
	\oldsubsection}% Regular \subsection

\newcommand{\abs}[1]{\left\lvert#1\right\rvert}
\newcommand{\norm}[1]{\left\lVert#1\right\rVert}

\newcommand{\g}{\text{g}}
\newcommand{\m}{\text{m}}
\newcommand{\cm}{\text{cm}}
\newcommand{\mm}{\text{mm}}
\newcommand{\s}{\text{s}}
\newcommand{\N}{\text{N}}
\newcommand{\Hz}{\text{Hz}}

\newcommand{\virgolette}[1]{``\text{#1}"}
\newcommand{\tildetext}{\raise.17ex\hbox{$\scriptstyle\mathtt{\sim}$}}


\renewcommand{\arraystretch}{1.2}

\addto\captionsenglish{\renewcommand{\figurename}{Fig.}}
\addto\captionsenglish{\renewcommand{\tablename}{Tab.}}

\DeclareCaptionLabelFormat{andtable}{#1~#2  \&  \tablename~\thetable}

%%%%%%%%%%%%%%%%%%%%%%%%%%%%%%%%%%%%%%%%%%%%%%%%%%%%%%%%%%%%%%%

%opening
\title{%
    \Huge Misura indiretta della velocità della luce \\
    \Large C.d.L. in Fisica, a.a. 2023-2024 \\ Università degli Studi di Milano}
\author{\LARGE Lucrezia Bioni, Leonardo Cerasi, Giulia Federica Bianca Coppi}
\date{19 ottobre 2023}

\begin{document}

    \maketitle

    \section{Analisi Dati}

    \subsection{Elaborazione Dati}

    A partire dalle coppie di misurazioni della frequenza $ \nu $ e della deviazione $ \delta $ per ciascuna configurazione, abbiamo calcolato la velocità angolare $ \omega $ e le rispettive variazioni $ \Delta\omega $ e $ \Delta\delta $ al variare della frequenza tramite le relazioni $ riferimenti-alle-equazioni $. I dati così elaborati sono riportati in $ riferimenti-alle-tabelle $. \\
    Una volta ottenute ciascuna coppia $ \left(\Delta\omega,\Delta\delta\right) $ abbiamo ottenuto i valori della velocità della luce tramite l'equazione $ riferimento-all-equazione $, i quali sono riportati in $ riferimento-alla-tabella $.
    
    \subsection{Stima degli errori}

    \begin{table}[!h]

        \centering
        \begin{tabular}{||c|c|c||}

            \hline
            $\nu_0 \,[\text{Hz}]$ & $\omega_0 \,[\text{rad/s}]$ \\
            \hline\hline
            $-11$ & $-69.11503838$ \\\hline
            $-10$ & $-62.83185307$ \\\hline
            $-10$ & $-62.83185307$ \\\hline
            $-11$ & $-69.11503838$ \\\hline
            $-15$ & $-94.24777961$ \\\hline
            $-15$ & $-94.24777961$ \\\hline
            $-17$ & $-106.8141502$ \\\hline
            $-16$ & $-100.5309649$ \\\hline
            $-18$ & $-113.0973355$ \\\hline
            $-18$ & $-113.0973355$ \\\hline
            $-18$ & $-113.0973355$ \\\hline
            $-18$ & $-113.0973355$ \\\hline
            $-18$ & $-113.0973355$ \\\hline
            $-18$ & $-113.0973355$ \\\hline
            $-18$ & $-113.0973355$ \\\hline
            $-18$ & $-113.0973355$ \\\hline
            $-18$ & $-113.0973355$ \\\hline
            $-18$ & $-113.0973355$ \\\hline
            $-18$ & $-113.0973355$ \\\hline
            $-18$ & $-113.0973355$ \\\hline
            $-18$ & $-113.0973355$ \\\hline
            $-13$ & $-81.68140899$ \\\hline
            $-17$ & $-106.8141502$ \\\hline
            $-18$ & $-113.0973355$ \\\hline
            $-18$ & $-113.0973355$ \\\hline
            $-18$ & $-113.0973355$ \\\hline
            $-18$ & $-113.0973355$ \\\hline
            $-18$ & $-113.0973355$ \\\hline
            $-18$ & $-113.0973355$ \\\hline
            $-18$ & $-113.0973355$ \\\hline
            
        \end{tabular}
        \caption{prova $\nu\delta$.}

    \end{table}

    \subsection{Confronto}

\end{document}