\documentclass[]{article}
\usepackage[utf8]{inputenc}
\usepackage[english]{babel}

% imports %%%%%%%%%%%%%%%%%%%%%%%%%%%%%%%%%%%%%%%%%%%%%%%%%%%%%%

\usepackage[]{csvsimple}

\usepackage{ragged2e}
\usepackage[left=25mm, right=25mm, top=15mm]{geometry}
\geometry{a4paper}
\usepackage{graphicx}
\usepackage{booktabs}
\usepackage{paralist}
\usepackage{subfig} 
\usepackage{fancyhdr}
\usepackage{amsmath}
\usepackage{amssymb}
\usepackage{amsfonts}
\usepackage{amsthm}
\usepackage{mathtools}
\usepackage{enumitem}
\usepackage{titlesec}
\usepackage{braket}
\usepackage{gensymb}
\usepackage{url}
\usepackage{hyperref}
\usepackage{csquotes}
\usepackage{multicol}
\usepackage{graphicx}
\usepackage{wrapfig}
\usepackage{babel}
\usepackage{caption}
\captionsetup{font=small}
\pagestyle{fancy}
\renewcommand{\headrulewidth}{0pt}
\lhead{}\chead{}\rhead{}
\lfoot{}\cfoot{\thepage}\rfoot{}
\usepackage{sectsty}
\usepackage[nottoc,notlof,notlot]{tocbibind}
\usepackage[titles,subfigure]{tocloft}
\renewcommand{\cftsecfont}{\rmfamily\mdseries\upshape}
\renewcommand{\cftsecpagefont}{\rmfamily\mdseries\upshape}

\let\oldsection\section% Store \section
\renewcommand{\section}{% Update \section
	\renewcommand{\theequation}{\thesection.\arabic{equation}}% Update equation number
	\oldsection}% Regular \section
\let\oldsubsection\subsection% Store \subsection
\renewcommand{\subsection}{% Update \subsection
	\renewcommand{\theequation}{\thesubsection.\arabic{equation}}% Update equation number
	\oldsubsection}% Regular \subsection

\newcommand{\abs}[1]{\left\lvert#1\right\rvert}
\newcommand{\norm}[1]{\left\lVert#1\right\rVert}

\newcommand{\g}{\text{g}}
\newcommand{\m}{\text{m}}
\newcommand{\cm}{\text{cm}}
\newcommand{\mm}{\text{mm}}
\newcommand{\s}{\text{s}}
\newcommand{\N}{\text{N}}
\newcommand{\Hz}{\text{Hz}}

\newcommand{\virgolette}[1]{``\text{#1}"}
\newcommand{\tildetext}{\raise.17ex\hbox{$\scriptstyle\mathtt{\sim}$}}


\renewcommand{\arraystretch}{1.2}

\addto\captionsenglish{\renewcommand{\figurename}{Fig.}}
\addto\captionsenglish{\renewcommand{\tablename}{Tab.}}

\DeclareCaptionLabelFormat{andtable}{#1~#2  \&  \tablename~\thetable}

%%%%%%%%%%%%%%%%%%%%%%%%%%%%%%%%%%%%%%%%%%%%%%%%%%%%%%%%%%%%%%%

%opening
\title{%
    \Huge Misura indiretta della velocità della luce \\
    \Large C.d.L. in Fisica, a.a. 2023-2024 \\ Università degli Studi di Milano}
\author{\LARGE Lucrezia Bioni, Leonardo Cerasi, Giulia Federica Bianca Coppi}
\date{19 ottobre 2023}

\begin{document}

    \maketitle

    \section{ Prova}

    Lorem ipsum $ \vec{F} = \oint m a $.

    $$
    \vec{F} = \oint m a
    $$

    \begin{equation}
        \vec{F} = \oint m a
    \end{equation}
    \begin{equation}
        \vec{F} = \oint m a
    \end{equation}

    Per quanto riguarda l'analisi del moto oscillatorio, prima di tutto è stata associata un'incertezza alle misure della posizione: per ogni posizione $h_i$ una buona stima dell'incertezza è data dall'intervallo che essa percorrerebbe nel tempo che intercorre tra due misurazioni del photogate nel caso ideale di moto non smorzato, quindi $\Delta h_i = v_i \Delta t = \frac{v_i}{\nu}$ e $\sigma_{h_i} = \frac{1}{2} \Delta h_i = \frac{v_i}{2\nu}$; per compensare la sottostima dell'errore che si verifica per valori piccoli della velocità, ovvero agli estremi
    
    \subsection{ciao}

    \subsubsection*{ciao}

    del moto, abbiamo preso come incertezza uguale per tutti gli $h_i$ la massima incertezza associabile con questo metodo, ovverosia quella corrispondente al valore massimo di velocità (non smorzata), quindi $v_m = \omega A$ con $A$ ampiezza dell'oscillazione (già stimata nella sezione 4 sulle Misure) e $\omega = \frac{2\pi}{T}$ pulsazione del moto ottenibile dal periodo. \\
    Abbiamo allora che $\sigma_h = \frac{\pi A}{\nu T}$, quindi che rimane soltanto da stimare il periodo d'oscillazione. Per fare ciò, abbiamo considerato il set di dati dei massimi d'oscillazioni, il set dei minimi e quello di massimi e minimi insieme; per i primi due sono stati calcolati i multipli del periodo (opportunamente divisi dal rispettivo numero di periodi intercorsi) associati a tutte le possibili combinazioni di coppie di massimi o minimi, per poi calcolare sia $T_{max}$ che $T_{min}$ come la media campionaria di questi periodi ottenuti, associando come incertezze $\sigma_{T_{max}}$ e $\sigma_{T_{min}}$ le corrispondenti deviazioni standard; per il set di dati contenente sia massimi che minimi, abbiamo ripetuto lo stesso procedimento calcolando però tutti i possibili mezzi periodi, ottenendo quindi $\tau = \frac{1}{2}T_{mm}$ e $\sigma_\tau = \frac{1}{2}\sigma_{T_{mm}}$. I valori così ottenuti sono:

    \begin{tabular}{||c|c|c||}

        \hline
        \bfseries Colonna 1 & \bfseries Colonna 2 & \bfseries Colonna 3
        \csvreader[head to column names]{prova.csv}{}
        {\\\hline\x & \y & \z} \\
        \hline
        
    \end{tabular}


\end{document}