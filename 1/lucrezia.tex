\documentclass{article} % tipologia di documento

\usepackage[utf8]{inputenc}
\usepackage[english]{babel}

% imports %%%%%%%%%%%%%%%%%%%%%%%%%%%%%%%%%%%%%%%%%%%%%%%%%%%%%%

\usepackage[]{csvsimple}

\usepackage{ragged2e}
\usepackage[left=25mm, right=25mm, top=15mm]{geometry}
\geometry{a4paper}
\usepackage{graphicx}
\usepackage{booktabs}
\usepackage{paralist}
\usepackage{subfig} 
\usepackage{fancyhdr}
\usepackage{amsmath} % Per la modalità matematica e il comando \bfseries
\usepackage{amssymb}
\usepackage{amsfonts}
\usepackage{amsthm}
\usepackage{mathtools}
\usepackage{enumitem}
\usepackage{titlesec}
\usepackage{braket}
\usepackage{gensymb}
\usepackage{url}
\usepackage{hyperref}
\usepackage{csquotes}
\usepackage{multicol}
\usepackage{graphicx}
\usepackage{wrapfig}
\usepackage{babel}
\usepackage{caption}
\usepackage{csvsimple} % Per leggere dati da file CSV
\usepackage{tabularx} 



\captionsetup{font=small}
\pagestyle{fancy}
\renewcommand{\headrulewidth}{0pt}
\lhead{}\chead{}\rhead{}
\lfoot{}\cfoot{\thepage}\rfoot{}
\usepackage{sectsty}
\usepackage[nottoc,notlof,notlot]{tocbibind}
\usepackage[titles,subfigure]{tocloft}
\renewcommand{\cftsecfont}{\rmfamily\mdseries\upshape}
\renewcommand{\cftsecpagefont}{\rmfamily\mdseries\upshape}

\let\oldsection\section% Store \section
\renewcommand{\section}{% Update \section
	\renewcommand{\theequation}{\thesection.\arabic{equation}}% Update equation number
	\oldsection}% Regular \section
\let\oldsubsection\subsection% Store \subsection
\renewcommand{\subsection}{% Update \subsection
	\renewcommand{\theequation}{\thesubsection.\arabic{equation}}% Update equation number
	\oldsubsection}% Regular \subsection

\newcommand{\abs}[1]{\left\lvert#1\right\rvert}
\newcommand{\norm}[1]{\left\lVert#1\right\rVert}

\newcommand{\g}{\text{g}}
\newcommand{\m}{\text{m}}
\newcommand{\cm}{\text{cm}}
\newcommand{\mm}{\text{mm}}
\newcommand{\s}{\text{s}}
\newcommand{\N}{\text{N}}
\newcommand{\Hz}{\text{Hz}}

\newcommand{\virgolette}[1]{``\text{#1}"}
\newcommand{\tildetext}{\raise.17ex\hbox{$\scriptstyle\mathtt{\sim}$}}


\renewcommand{\arraystretch}{1.2}

\addto\captionsenglish{\renewcommand{\figurename}{Fig.}}
\addto\captionsenglish{\renewcommand{\tablename}{Tab.}}

\DeclareCaptionLabelFormat{andtable}{#1~#2  \&  \tablename~\thetable}


\title{Il mio piccolo contributo} % titolo del documento
\author{Lucrezia Bioni (13655A)} % autore del documento
\date{} % data: se è vuoto non mette nulla :)

\begin{document} % inizia il documento
    \maketitle
    \section{Misure} % tra {} titolo del paragrafo
    
    Si è inizialmente misurata la distanza $D$ tra lo specchio rotante e lo specchio concavo. La distanza $D$ è stata misurata 
    mediante un metro di risoluzione $ 0,01 \, \text{m} $, la quale è stata attribuita come incertezza al valore della misura.

    \begin{equation}
        \label{equation for D}
        D = (13.28 \pm 0.01) \, \text{m}
    \end{equation} \\
    Si è poi misurata la distanza $A$ tra la seconda lente dell'apparato e lo specchio rotante. La misura è stata effettuata
    attraverso un calibro di risoluzione $ 0,01 \, \text{m} $, che è stata attribuita come incertezza di $a$.
    
    \begin{equation}
        \label{equation for a}
        a = (0.474 \pm 0.001) \, \text{m}
    \end{equation} \\
    Dopo aver avviato lo specchio rotante in senso orario a una frequenza di rotazione $ \nu_0 $ nell'intervallo $[10,20] \, \text{Hz}$,
    si è misurata, mediante micrometro di risoluzione $ 0.00001 \text{m} $ , la posizione dello spot luminoso $\delta_0$ visibile attraverso
    un microscopio. Si è poi portato lo specchio in un intorno della frequenza massima di rotazione $ \nu $ e si è misurata,
    sempre mediante micrometro, la nuova posizione dello spot luminoso $\delta$. Tale set di misure è stato ripetuto per $30$ volte.
    I dati rilevati sono riportati nella tabella $ riferimento-alla-tabella $. \\ 

    Si sono prese le medesime misure di posizione, a frequenza minima e a frequenza massima, facendo ruotare
    lo specchio in senso antiorario. I dati rilevati sono riportati nella tabella $ riferimento-alla-tabella $. \\ 

    Si sono poi effettuate misure di posizione dello spot luminoso portando lo specchio dalla frequenza massima di rotazione in senso
    orario $\nu_0$ alla frequenza massima di rotazione in senso antiorario $\nu$.
    I dati rilevati sono riportati nella tabella $ riferimento-alla-tabella $. \\

    Infine, si sono rilevate misure di posizione dello spot luminoso con lo specchio a frequenza minima di rotazione ( $[10,20] \, \text{Hz}$ )
    e a frequenze intermedie, sia in senso orario (i dati sono riportati nella tabella $ riferimento-alla-tabella $) sia 
    in senso antiorario (i dati sono riportati nella tabella $ riferimento-alla-tabella $). \\
    
    \section {Analisi dati}


    \subsection {Stima degli errori}
    %% finisco parte iniziata da Giulia

    L'errore sulla stima del valore di $ c $, $ \sigma_c(tot) $, è frutto di due componenti, una statistica $\sigma_c(stat)$ e una sistematica $\sigma_c(sist)$:
    \begin{equation}
        \label{sigma_tot}
        \sigma_c(tot) = \sqrt{ \sigma_c(stat) ^2 + \sigma_c(sist) ^2 } 
    \end{equation}
    
    La componente statistica viene determinata attraverso la deviazione standard della media dei valori ottenuti per $c$ per ogni set di misure di $ \Delta \omega $ e $ \Delta \delta $. \\
    La componente sistematica viene determinata attraverso la propagazione degli errori sulle misure delle grandezze $D$ e $a$ nella formula per la determinazione della velocità della luce $ riferimento-alla-formula $. In tale procedimento si considerano le grandezze  $\omega$ e $\delta$ come prive di errore, poiché questo è già stato considerato nella componente statistica dell'errore.

    \begin{equation}
        \label{sigma_stat}
        \sigma_c(stat) = \sqrt{ (\frac{\partial c}{\partial D})^2 \, \sigma_D ^2 + (\frac{\partial c}{\partial a})^2 \, \sigma_a ^2 } 
    \end{equation}

    dove

    \begin{equation}
        \label{eqn:propD}
        \frac{\partial c}{\partial D} = \frac{4\Delta \omega}{\Delta \delta} \frac{D f_2 (2a + D -2f_2)}{(a + D -f_2)^2}
        \end{equation}

    \begin{equation}
        \label{eqn:propa}
        \frac{\partial c}{\partial a} = -\frac{4\Delta \omega}{\Delta \delta} \frac{D^2 f_2}{(a + D -f_2)^2}
        \end{equation}

    \section{Appendice}

    \begin{table}
        \centering
        \begin{tabular}{||c|c|c||c|c|c||}
            \hline
            $\nu_0 \, [\text{Hz}]$ & $\omega_0 \, [\text{rad/s}]$ &  $\delta_0 \,[\text{m}]$ &  $\nu \,[\text{Hz}]$ & $\omega \,[\text{rad/s}]$ & $\delta \, [\text{m}]$ \\
            \hline\hline
            $-11 $ & $-69.11503838 $ & $ 9.32 $ & $ -1390 $ & $ -8733.627577 $ & $  8.93 $ \\\hline
            $-10 $ & $-62.83185307 $ & $ 9.31 $ & $ -1317 $ & $ -8274.955050 $ & $  8.93 $ \\\hline
            $-10 $ & $-62.83185307 $ & $ 9.31 $ & $ -1355 $ & $ -8513.716091 $ & $  8.94 $ \\\hline
            $-11 $ & $-69.11503838 $ & $ 9.30 $ & $ -1404 $ & $ -8821.592171 $ & $  8.93 $ \\\hline
            $-15 $ & $-94.24777961 $ & $ 9.31 $ & $ -1359 $ & $ -8538.848832 $ & $  8.94 $ \\\hline
            $-15 $ & $-94.24777961 $ & $ 9.31 $ & $ -1409 $ & $ -8853.008098 $ & $  8.92 $ \\\hline
            $-17 $ & $-106.8141502 $ & $ 9.30 $ & $ -1374 $ & $ -8633.096612 $ & $  8.95 $ \\\hline
            $-16 $ & $-100.5309649 $ & $ 9.30 $ & $ -1386 $ & $ -8708.494836 $ & $  8.94 $ \\\hline
            $-18 $ & $-113.0973355 $ & $ 9.31 $ & $ -1386 $ & $ -8708.494836 $ & $  8.92 $ \\\hline
            $-18 $ & $-113.0973355 $ & $ 9.31 $ & $ -1409 $ & $ -8853.008098 $ & $  8.92 $ \\\hline
            $-18 $ & $-113.0973355 $ & $ 9.31 $ & $ -1363 $ & $ -8563.981574 $ & $  8.94 $ \\\hline
            $-18 $ & $-113.0973355 $ & $ 9.30 $ & $ -1143 $ & $ -7181.680806 $ & $  8.96 $ \\\hline
            $-18 $ & $-113.0973355 $ & $ 9.31 $ & $ -1441 $ & $ -9054.070028 $ & $  8.97 $ \\\hline
            $-18 $ & $-113.0973355 $ & $ 9.31 $ & $ -1450 $ & $ -9110.618695 $ & $  8.92 $ \\\hline
            $-18 $ & $-113.0973355 $ & $ 9.33 $ & $ -1414 $ & $ -8884.424024 $ & $  8.94 $ \\\hline
            $-18 $ & $-113.0973355 $ & $ 9.32 $ & $ -1401 $ & $ -8802.742615 $ & $  8.95 $ \\\hline
            $-18 $ & $-113.0973355 $ & $ 9.31 $ & $ -1410 $ & $ -8859.291283 $ & $  8.93 $ \\\hline
            $-18 $ & $-113.0973355 $ & $ 9.31 $ & $ -1431 $ & $ -8991.238175 $ & $  8.93 $ \\\hline
            $-18 $ & $-113.0973355 $ & $ 9.31 $ & $ -1444 $ & $ -9072.919584 $ & $  8.93 $ \\\hline
            $-18 $ & $-113.0973355 $ & $ 9.30 $ & $ -1424 $ & $ -8947.255877 $ & $  8.94 $ \\\hline
            $-18 $ & $-113.0973355 $ & $ 9.30 $ & $ -1395 $ & $ -8765.043504 $ & $  8.92 $ \\\hline
            $-13 $ & $-81.68140899 $ & $ 9.30 $ & $ -1455 $ & $ -9142.034622 $ & $  8.92 $ \\\hline
            $-17 $ & $-106.8141502 $ & $ 9.31 $ & $ -1456 $ & $ -9148.317807 $ & $  8.94 $ \\\hline
            $-18 $ & $-113.0973355 $ & $ 9.30 $ & $ -1469 $ & $ -9229.999216 $ & $  8.91 $ \\\hline
            $-18 $ & $-113.0973355 $ & $ 9.30 $ & $ -1426 $ & $ -8959.822248 $ & $  8.92 $ \\\hline
            $-18 $ & $-113.0973355 $ & $ 9.31 $ & $ -1446 $ & $ -9085.485954 $ & $  8.92 $ \\\hline
            $-18 $ & $-113.0973355 $ & $ 9.31 $ & $ -1418 $ & $ -8909.556766 $ & $  8.92 $ \\\hline
            $-18 $ & $-113.0973355 $ & $ 9.30 $ & $ -1446 $ & $ -9085.485954 $ & $  8.90 $ \\\hline
            $-18 $ & $-113.0973355 $ & $ 9.31 $ & $ -1455 $ & $ -9142.034622 $ & $  8.91 $ \\\hline
            $-18 $ & $-113.0973355 $ & $ 9.31 $ & $ -1446 $ & $ -9085.485954 $ & $  8.93 $ \\\hline
        \end{tabular}
        \caption{Specchio in rotazione CW a frequenza iniziale minima $\nu_0$ e frequenza finale massima $\nu$: misure di posizione iniziale $\delta_0$ e finale $\delta$ dello spot luminoso}
        \label{CW_min_max}
    \end{table}


    \begin{table}
        \centering
        \begin{tabular}{||c|c|c||c|c|c||}
            \hline
            $\nu_0 \, [\text{Hz}]$ & $\omega_0 \, [\text{rad/s}]$ &  $\delta_0 \,[\text{m}]$ &  $\nu \,[\text{Hz}]$ & $\omega \,[\text{rad/s}]$ & $\delta \, [\text{m}]$ \\
            \hline\hline
            $17$ & $106.8141502$ & $9.31$ & $1400$ & $8796.459430$ & $9.70$ \\\hline
            $18$ & $113.0973355$ & $9.31$ & $1421$ & $8928.406322$ & $9.70$ \\\hline
            $18$ & $113.0973355$ & $9.31$ & $1422$ & $8934.689507$ & $9.69$ \\\hline
            $18$ & $113.0973355$ & $9.31$ & $1393$ & $8752.477133$ & $9.71$ \\\hline
            $18$ & $113.0973355$ & $9.32$ & $1399$ & $8790.176245$ & $9.70$ \\\hline
            $17$ & $106.8141502$ & $9.31$ & $1396$ & $8771.326689$ & $9.70$ \\\hline
            $18$ & $113.0973355$ & $9.32$ & $1414$ & $8884.424024$ & $9.69$ \\\hline
            $18$ & $113.0973355$ & $9.32$ & $1391$ & $8739.910762$ & $9.69$ \\\hline
            $18$ & $113.0973355$ & $9.32$ & $1376$ & $8645.662983$ & $9.69$ \\\hline
            $17$ & $106.8141502$ & $9.32$ & $1404$ & $8821.592171$ & $9.69$ \\\hline
            $18$ & $113.0973355$ & $9.32$ & $1434$ & $9010.087730$ & $9.70$ \\\hline
            $17$ & $106.8141502$ & $9.31$ & $1334$ & $8381.769200$ & $9.71$ \\\hline
            $18$ & $113.0973355$ & $9.32$ & $1342$ & $8432.034682$ & $9.70$ \\\hline
            $17$ & $106.8141502$ & $9.32$ & $1363$ & $8563.981574$ & $9.69$ \\\hline
            $18$ & $113.0973355$ & $9.32$ & $1317$ & $8274.955050$ & $9.70$ \\\hline
            $18$ & $113.0973355$ & $9.32$ & $1351$ & $8488.583350$ & $9.70$ \\\hline
            $18$ & $113.0973355$ & $9.31$ & $1316$ & $8268.671864$ & $9.70$ \\\hline
            $17$ & $106.8141502$ & $9.31$ & $1330$ & $8356.636459$ & $9.70$ \\\hline
            $18$ & $113.0973355$ & $9.31$ & $1380$ & $8670.795724$ & $9.71$ \\\hline
            $18$ & $113.0973355$ & $9.32$ & $1444$ & $9072.919584$ & $9.71$ \\\hline
            $18$ & $113.0973355$ & $9.32$ & $1435$ & $9016.370916$ & $9.70$ \\\hline
            $17$ & $106.8141502$ & $9.32$ & $1359$ & $8538.848832$ & $9.70$ \\\hline
            $18$ & $113.0973355$ & $9.32$ & $1378$ & $8658.229353$ & $9.70$ \\\hline
            $18$ & $113.0973355$ & $9.32$ & $1412$ & $8871.857654$ & $9.70$ \\\hline
            $17$ & $106.8141502$ & $9.31$ & $1378$ & $8658.229353$ & $9.67$ \\\hline
            $18$ & $113.0973355$ & $9.31$ & $1424$ & $8947.255877$ & $9.71$ \\\hline
            $18$ & $113.0973355$ & $9.32$ & $1438$ & $9035.220472$ & $9.70$ \\\hline
            $18$ & $113.0973355$ & $9.31$ & $1421$ & $8928.406322$ & $9.69$ \\\hline
            $18$ & $113.0973355$ & $9.31$ & $1423$ & $8940.972692$ & $9.69$ \\\hline
            $18$ & $113.0973355$ & $9.31$ & $1421$ & $8928.406322$ & $9.69$ \\\hline
        \end{tabular}
        \caption{Specchio in rotazione CCW a frequenza iniziale minima $\nu_0$ e frequenza finale massima $\nu$: misure di posizione iniziale $\delta_0$ e finale $\delta$ dello spot luminoso}
        \label{CCW_min_max}
    \end{table}



    \begin{table}
        \centering
        \begin{tabular}{||c|c|c||}
            \hline
            $\Delta \omega [\text{rad/s}]$ & $\Delta \delta [\text{m}]$ & $c [\text{m/s}]$ \\
            \hline\hline
            $-8664.512539$ & $-0.39$ & $2.92508E+08$ \\\hline
            $-8212.123196$ & $-0.38$ & $2.84532E+08$ \\\hline
            $-8450.884238$ & $-0.37$ & $3.00718E+08$ \\\hline
            $-8752.477133$ & $-0.37$ & $3.11450E+08$ \\\hline
            $-8444.601053$ & $-0.37$ & $3.00494E+08$ \\\hline
            $-8758.760318$ & $-0.39$ & $2.95690E+08$ \\\hline
            $-8526.282462$ & $-0.35$ & $3.20738E+08$ \\\hline
            $-8607.963871$ & $-0.36$ & $3.14816E+08$ \\\hline
            $-8595.397000$ & $-0.39$ & $2.90175E+08$ \\\hline
            $-8739.910762$ & $-0.39$ & $2.95054E+08$ \\\hline
            $-8450.884238$ & $-0.37$ & $3.00718E+08$ \\\hline
            $-7068.583471$ & $-0.34$ & $2.73723E+08$ \\\hline
            $-8940.972692$ & $-0.34$ & $3.46230E+08$ \\\hline
            $-8997.521360$ & $-0.39$ & $3.03750E+08$ \\\hline
            $-8771.326689$ & $-0.39$ & $2.96114E+08$ \\\hline
            $-8689.645280$ & $-0.37$ & $3.09214E+08$ \\\hline
            $-8746.193948$ & $-0.38$ & $3.03036E+08$ \\\hline
            $-8878.140839$ & $-0.38$ & $3.07608E+08$ \\\hline
            $-8959.822248$ & $-0.38$ & $3.10438E+08$ \\\hline
            $-8834.158542$ & $-0.36$ & $3.23088E+08$ \\\hline
            $-8651.946168$ & $-0.38$ & $2.99770E+08$ \\\hline
            $-9060.353213$ & $-0.38$ & $3.13921E+08$ \\\hline
            $-9041.503657$ & $-0.37$ & $3.21734E+08$ \\\hline
            $-9116.901881$ & $-0.39$ & $3.07781E+08$ \\\hline
            $-8846.724913$ & $-0.38$ & $3.06519E+08$ \\\hline
            $-8972.388619$ & $-0.39$ & $3.02902E+08$ \\\hline
            $-8796.459430$ & $-0.39$ & $2.96963E+08$ \\\hline
            $-8972.388619$ & $-0.40$ & $2.95329E+08$ \\\hline
            $-9028.937286$ & $-0.40$ & $2.97191E+08$ \\\hline
            $-8972.388619$ & $-0.38$ & $3.10873E+08$ \\\hline
        \end{tabular}
        \caption{Specchio in rotazione CW, frequenza iniziale minima e frequenza finale massima: variazione di pulsazione $\Delta\omega$ e variazione di posizione $\Delta\delta$, e rispettiva misura indiretta della velocità della luce $c$}
        \label{CW_min_max}
    \end{table}


    \begin{table}
        \centering
        \begin{tabular}{||c|c|c||}
            \hline
            $\Delta \omega [\text{rad/s}]$ & $\Delta \delta [\text{m}]$ & $c [\text{m/s}]$ \\
            \hline\hline
            $8689.645280$ & $0.39$ & $2.93357E+08$ \\\hline
            $8815.308986$ & $0.39$ & $2.97599E+08$ \\\hline
            $8821.592171$ & $0.38$ & $3.05648E+08$ \\\hline
            $8639.379797$ & $0.40$ & $2.84368E+08$ \\\hline
            $8677.078909$ & $0.38$ & $3.00641E+08$ \\\hline
            $8664.512539$ & $0.39$ & $2.92508E+08$ \\\hline
            $8771.326689$ & $0.37$ & $3.12120E+08$ \\\hline
            $8626.813427$ & $0.37$ & $3.06978E+08$ \\\hline
            $8532.565647$ & $0.37$ & $3.03624E+08$ \\\hline
            $8714.778021$ & $0.37$ & $3.10108E+08$ \\\hline
            $8896.990395$ & $0.38$ & $3.08261E+08$ \\\hline
            $8274.955050$ & $0.40$ & $2.72373E+08$ \\\hline
            $8318.937347$ & $0.38$ & $2.88232E+08$ \\\hline
            $8457.167423$ & $0.37$ & $3.00941E+08$ \\\hline
            $8161.857714$ & $0.38$ & $2.82790E+08$ \\\hline
            $8375.486014$ & $0.38$ & $2.90192E+08$ \\\hline
            $8155.574529$ & $0.39$ & $2.75327E+08$ \\\hline
            $8249.822308$ & $0.39$ & $2.78509E+08$ \\\hline
            $8557.698388$ & $0.40$ & $2.81680E+08$ \\\hline
            $8959.822248$ & $0.39$ & $3.02478E+08$ \\\hline
            $8903.273580$ & $0.38$ & $3.08478E+08$ \\\hline
            $8432.034682$ & $0.38$ & $2.92151E+08$ \\\hline
            $8545.132018$ & $0.38$ & $2.96070E+08$ \\\hline
            $8758.760318$ & $0.38$ & $3.03471E+08$ \\\hline
            $8551.415203$ & $0.36$ & $3.12748E+08$ \\\hline
            $8834.158542$ & $0.40$ & $2.90779E+08$ \\\hline
            $8922.123136$ & $0.38$ & $3.09131E+08$ \\\hline
            $8815.308986$ & $0.38$ & $3.05431E+08$ \\\hline
            $8827.875357$ & $0.38$ & $3.05866E+08$ \\\hline
            $8815.308986$ & $0.38$ & $3.05431E+08$ \\\hline
        \end{tabular}
        \caption{Specchio in rotazione CCW, frequenza iniziale minima e frequenza finale massima: variazione di pulsazione $\Delta\omega$ e variazione di posizione $\Delta\delta$, e rispettiva misura indiretta della velocità della luce $c$}
        \label{CCW_min_max}
    \end{table}


    \begin{table}
        \centering
        \begin{tabular}{||c|c|c||}
            \hline
            $\Delta \omega [\text{rad/s}]$ & $\Delta \delta [\text{m}]$ & $c [\text{m/s}]$ \\
            \hline\hline
            $17479.82152$ & $0.76$ & $3.02818E+08$ \\\hline
            $17630.61797$ & $0.77$ & $3.01464E+08$ \\\hline
            $17008.58263$ & $0.77$ & $2.90828E+08$ \\\hline
            $17618.05160$ & $0.79$ & $2.93623E+08$ \\\hline
            $17228.49411$ & $0.77$ & $2.94588E+08$ \\\hline
            $17266.19322$ & $0.78$ & $2.91448E+08$ \\\hline
            $17291.32597$ & $0.78$ & $2.91872E+08$ \\\hline
            $17668.31708$ & $0.79$ & $2.94460E+08$ \\\hline
            $17335.30826$ & $0.77$ & $2.96414E+08$ \\\hline
            $17812.83035$ & $0.79$ & $2.96869E+08$ \\\hline
            $17706.01620$ & $0.76$ & $3.06737E+08$ \\\hline
            $17498.67108$ & $0.78$ & $2.95372E+08$ \\\hline
            $17724.86575$ & $0.79$ & $2.95403E+08$ \\\hline
            $16549.91010$ & $0.74$ & $2.94457E+08$ \\\hline
            $17511.23745$ & $0.76$ & $3.03362E+08$ \\\hline
            $16926.90122$ & $0.76$ & $2.93239E+08$ \\\hline
            $16920.61803$ & $0.74$ & $3.01053E+08$ \\\hline
            $17222.21093$ & $0.74$ & $3.06419E+08$ \\\hline
            $17398.14012$ & $0.77$ & $2.97489E+08$ \\\hline
            $16908.05166$ & $0.70$ & $3.18020E+08$ \\\hline
            $16700.70655$ & $0.76$ & $2.89321E+08$ \\\hline
            $17266.19322$ & $0.77$ & $2.95233E+08$ \\\hline
            $17002.29944$ & $0.73$ & $3.06650E+08$ \\\hline
            $16688.14018$ & $0.74$ & $2.96917E+08$ \\\hline
            $17146.81270$ & $0.74$ & $3.05078E+08$ \\\hline
            $17247.34367$ & $0.77$ & $2.94910E+08$ \\\hline
            $17335.30826$ & $0.78$ & $2.92614E+08$ \\\hline
            $17259.91004$ & $0.74$ & $3.07090E+08$ \\\hline
            $16870.35255$ & $0.76$ & $2.92260E+08$ \\\hline
            $16788.67114$ & $0.75$ & $2.94723E+08$ \\\hline
        \end{tabular}
        \caption{Specchio in rotazione iniziale CW e rotazione finale CCW: variazione di pulsazione $\Delta\omega$ e variazione di posizione $\Delta\delta$, e rispettiva misura indiretta della velocità della luce $c$}
        \label{CW_CCW}
    \end{table}

    \begin{table}
        \centering
        \begin{tabular}{||c|c|c||}
            \hline
            $\Delta \omega [\text{rad/s}]$ & $\Delta \delta [\text{m}]$ & $c [\text{m/s}]$ \\
            \hline\hline
            $-4209.734156$&$-0.20$&$2.77130E+08$\\\hline
            $-4995.132319$&$-0.21$&$3.13174E+08$\\\hline
            $-4517.610236$&$-0.21$&$2.83236E+08$\\\hline
            $-5296.725214$&$-0.23$&$3.03206E+08$\\\hline
            $-5328.141140$&$-0.24$&$2.92296E+08$\\\hline
            $-5177.344693$&$-0.23$&$2.96372E+08$\\\hline
            $-5334.424326$&$-0.24$&$2.92641E+08$\\\hline
            $-5284.158843$&$-0.24$&$2.89883E+08$\\\hline
            $-5152.211952$&$-0.22$&$3.08340E+08$\\\hline
            $-5510.353514$&$-0.24$&$3.02292E+08$\\\hline
            $-5196.194249$&$-0.23$&$2.97451E+08$\\\hline
            $-5158.495137$&$-0.22$&$3.08716E+08$\\\hline
            $-5409.822549$&$-0.23$&$3.09680E+08$\\\hline
            $-5453.804847$&$-0.25$&$2.87222E+08$\\\hline
            $-5460.088032$&$-0.23$&$3.12558E+08$\\\hline
        \end{tabular}
        \caption{Specchio in rotazione CW, frequenza iniziale minima e frequenza finale intermedia: variazione di pulsazione $\Delta\omega$ e variazione di posizione $\Delta\delta$, e rispettiva misura indiretta della velocità della luce $c$}
        \label{CW_min_mid}
    \end{table}

        %%%%%%%%%%%%%%%%%%
        %tabelle che mannaggia non funzionano


    %\begin{table}
    %    \centering
    %    \begin{tabular}{||c|c|c||c|c|c||}
%
    %        \hline
    %        \bfseries $\nu_0 \, [\text{Hz}]$ & \bfseries $\omega_0 \, [\text{rad/s}]$ & \bfseries $\delta_0 \,[\text{m}]$ & \bfseries $\nu \,[\text{Hz}]$ & \bfseries $\omega \,[\text{rad/s}]$ & \bfseries $\delta \, [\text{m}]$
    %        \csvreader[head to column names]{csv/CW_min_max.csv}{}
    %        {\\\hline\nuzero & \omegazero & \deltazero & \nu & \omega & \delta  } \\
    %        \hline
    %    
    %    \end{tabular}
    %    \caption{Misure delle freqeuenze iniziale e finale dello specchio in rotazione CW e misure di posizione iniziale e finale dello spot luminoso}
    %\end{table}
%
    % \begin{table}
    %    \centering
    %    \begin{tabular}{||c|c|c||c|c|c||}
%
    %        \hline
    %        \bfseries $ \nu_0 \, [\text{Hz}]$ & \bfseries $\omega_0 \, [\text{rad/s}]$ & \bfseries $\delta_0 \,[\text{m}]$ & \bfseries $\nu \,[\text{Hz}]$ & \bfseries $\omega \,[\text{rad/s}]$ & \bfseries $\delta \, [\text{m}]$
    %        \csvreader[head to column names]{csv/CCW_min_max.csv}{}
    %        {\\\hline\nu0 & \omegazero & \deltazero & \nu & \omega & \delta  } \\
    %        \hline
    %    
    %    \end{tabular}
    %    \caption{Misure delle freqeuenze iniziale e finale dello specchio in rotazione CCW e misure di posizione iniziale e finale dello spot luminoso}
    %\end{table}
%
%
    %\begin{table}
    %    \centering
    %    \begin{tabular}{||c|c|c||}
%
    %        \hline
    %        \bfseries $\Delta \omega [\text{rad/s}]$ & \bfseries $\Delta \delta [\text{m}]$ & \bfseries $c [\text{m/s}]$
    %        \csvreader[head to column names]{csv/CW_min_max.csv}{}
    %        {\\\hline \deltaomega & \deltadelta & \c} \\
    %        \hline
    %    
    %    \end{tabular}
    %    \caption{Specchio in rotazione CW: misure delle variazioni di pulsazione e delle variazioni di posizione, e della relativa velocità della luce}
    %\end{table}
%
%
    %\begin{table}
    %    \centering
    %    \begin{tabular}{||c|c|c||}
%
    %        \hline
    %        \bfseries $\Delta \omega [\text{rad/s}]$ & \bfseries $\Delta \delta [\text{m}]$ & \bfseries $c [\text{m/s}]$
    %        \csvreader[head to column names] {csv/CCW_min_max.csv}{}
    %        {\\\hline \deltaomega & \deltadelta & \c} \\
    %        \hline
%
    %    \end{tabular}
    %    \caption{Specchio in rotazione CCW: misure delle variazioni di pulsazione e delle variazioni di posizione, e della relativa velocità della luce}
    %\end{table}
%
%
%
%

\end{document} % fine il documento: non necessario perché lo farà automaticamente