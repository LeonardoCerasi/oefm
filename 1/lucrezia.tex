\documentclass{article} % tipologia di documento

\usepackage[utf8]{inputenc}
\usepackage[english]{babel}

% imports %%%%%%%%%%%%%%%%%%%%%%%%%%%%%%%%%%%%%%%%%%%%%%%%%%%%%%

\usepackage[]{csvsimple}

\usepackage{ragged2e}
\usepackage[left=25mm, right=25mm, top=15mm]{geometry}
\geometry{a4paper}
\usepackage{graphicx}
\usepackage{booktabs}
\usepackage{paralist}
\usepackage{subfig} 
\usepackage{fancyhdr}
\usepackage{amsmath} % Per la modalità matematica e il comando \bfseries
\usepackage{amssymb}
\usepackage{amsfonts}
\usepackage{amsthm}
\usepackage{mathtools}
\usepackage{enumitem}
\usepackage{titlesec}
\usepackage{braket}
\usepackage{gensymb}
\usepackage{url}
\usepackage{hyperref}
\usepackage{csquotes}
\usepackage{multicol}
\usepackage{graphicx}
\usepackage{wrapfig}
\usepackage{babel}
\usepackage{caption}
\usepackage{csvsimple} % Per leggere dati da file CSV
\usepackage{tabularx} 


\usepackage{array}  % Per la gestione delle tabelle
\usepackage{siunitx}  % Per formattare i numeri




\captionsetup{font=small}
\pagestyle{fancy}
\renewcommand{\headrulewidth}{0pt}
\lhead{}\chead{}\rhead{}
\lfoot{}\cfoot{\thepage}\rfoot{}
\usepackage{sectsty}
\usepackage[nottoc,notlof,notlot]{tocbibind}
\usepackage[titles,subfigure]{tocloft}
\renewcommand{\cftsecfont}{\rmfamily\mdseries\upshape}
\renewcommand{\cftsecpagefont}{\rmfamily\mdseries\upshape}

\let\oldsection\section% Store \section
\renewcommand{\section}{% Update \section
	\renewcommand{\theequation}{\thesection.\arabic{equation}}% Update equation number
	\oldsection}% Regular \section
\let\oldsubsection\subsection% Store \subsection
\renewcommand{\subsection}{% Update \subsection
	\renewcommand{\theequation}{\thesubsection.\arabic{equation}}% Update equation number
	\oldsubsection}% Regular \subsection

\newcommand{\abs}[1]{\left\lvert#1\right\rvert}
\newcommand{\norm}[1]{\left\lVert#1\right\rVert}

\newcommand{\g}{\text{g}}
\newcommand{\m}{\text{m}}
\newcommand{\cm}{\text{cm}}
\newcommand{\mm}{\text{mm}}
\newcommand{\s}{\text{s}}
\newcommand{\N}{\text{N}}
\newcommand{\Hz}{\text{Hz}}

\newcommand{\virgolette}[1]{``\text{#1}"}
\newcommand{\tildetext}{\raise.17ex\hbox{$\scriptstyle\mathtt{\sim}$}}


\renewcommand{\arraystretch}{1.2}

\addto\captionsenglish{\renewcommand{\figurename}{Fig.}}
\addto\captionsenglish{\renewcommand{\tablename}{Tab.}}

\DeclareCaptionLabelFormat{andtable}{#1~#2  \&  \tablename~\thetable}


\title{Il mio piccolo contributo} % titolo del documento
\author{Lucrezia Bioni (13655A)} % autore del documento
\date{} % data: se è vuoto non mette nulla :)

\begin{document} % inizia il documento
    \maketitle
    \section{Misure} % tra {} titolo del paragrafo
    
    Si è inizialmente misurata la distanza $D$ tra lo specchio rotante e lo specchio concavo. La distanza $D$ è stata misurata 
    mediante un metro di risoluzione $ 0,01 \, \text{m} $, la quale è stata attribuita come incertezza al valore della misura.

    \begin{equation}
        \label{equation for D}
        D = (13.28 \pm 0.01) \, \text{m}
    \end{equation} \\
    Si è poi misurata la distanza $A$ tra la seconda lente dell'apparato e lo specchio rotante. La misura è stata effettuata
    attraverso un calibro di risoluzione $ 0,01 \, \text{m} $, che è stata attribuita come incertezza di $a$.
    
    \begin{equation}
        \label{equation for a}
        a = (0.474 \pm 0.001) \, \text{m}
    \end{equation} \\
    Dopo aver avviato lo specchio rotante in senso orario a una frequenza di rotazione $ \nu_0 $ nell'intervallo $[10,20] \, \text{Hz}$,
    si è misurata, mediante micrometro di risoluzione $ 0.00001 \text{m} $ , la posizione dello spot luminoso $\delta_0$ visibile attraverso
    un microscopio. Si è poi portato lo specchio in un intorno della frequenza massima di rotazione $ \nu $ e si è misurata,
    sempre mediante micrometro, la nuova posizione dello spot luminoso $\delta$. Tale set di misure è stato ripetuto per $30$ volte.
    I dati rilevati sono riportati nella tabella $ riferimento-alla-tabella $. \\ 

    Si sono prese le medesime misure di posizione, a frequenza minima e a frequenza massima, facendo ruotare
    lo specchio in senso antiorario. I dati rilevati sono riportati nella tabella $ riferimento-alla-tabella $. \\ 

    Si sono poi effettuate misure di posizione dello spot luminoso portando lo specchio dalla frequenza massima di rotazione in senso
    orario $\nu_0$ alla frequenza massima di rotazione in senso antiorario $\nu$.
    I dati rilevati sono riportati nella tabella $ riferimento-alla-tabella $. \\

    Infine, si sono rilevate misure di posizione dello spot luminoso con lo specchio a frequenza minima di rotazione ( $[10,20] \, \text{Hz}$ )
    e a frequenze intermedie, sia in senso orario (i dati sono riportati nella tabella $ riferimento-alla-tabella $) sia 
    in senso antiorario (i dati sono riportati nella tabella $ riferimento-alla-tabella $). \\
    


    \begin{table}
        \centering
        \begin{tabular}{c|c|c|c|c|c}

            \hline
            \bfseries $\nu_0 \, [\text{Hz}]$ & \bfseries $\omega_0 \, [\text{rad/s}]$ & \bfseries $\delta_0 \,[\text{m}]$ & \bfseries $\nu \,[\text{Hz}]$ & \bfseries $\omega \,[\text{rad/s}]$ & \bfseries $\delta \, [\text{m}]$
            \csvreader[head to column names]{csv/CW_min_max.csv}{}
            {\\\hline\nuzero & \omegazero & \deltazero & \nu & \omega & \delta  } \\
            \hline
        
        \end{tabular}
        \caption{Specchio in rotazione CW a frequenze $\nu_0$ e $\nu$: misure di posizione $\delta_0$ e $\delta$ dello spot luminoso}
    \end{table}


    Tabella 1.2: Specchio in rotazione CCW a frequenze $\nu_0$ e $\nu$: misure di posizione $\delta_0$ e $\delta$ dello spot luminoso \\
    \begin{tabular}{c|c|c|c|c|c}

        \hline
        \bfseries $\nu_0 \, [\text{Hz}]$ & \bfseries $\omega_0 \, [\text{rad/s}]$ & \bfseries $\delta_0 \,[\text{m}]$ & \bfseries $\nu \,[\text{Hz}]$ & \bfseries $\omega \,[\text{rad/s}]$ & \bfseries $\delta \, [\text{m}]$
        \csvreader[head to column names]{csv/CCW_min_max.csv}{}
        {\\\hline\nu0 & \omegazero & \deltazero & \nu & \omega & \delta  } \\
        \hline
        
    \end{tabular}


    
    \begin{tabular}{c|c|c}

        \hline
        \bfseries $\Delta \omega $ & \bfseries $\Delta \delta $ & \bfseries $c$
        \csvreader[head to column names]{csv/CCW_min_max.csv}{}
        {\\\hline \deltaomega & \deltadelta & c} \\
        \hline
        
    \end{tabular}


  


    




\end{document} % fine il documento: non necessario perché lo farà automaticamente