\documentclass[]{article}
\usepackage[utf8]{inputenc}
\usepackage[english]{babel}

\usepackage{ragged2e}
\usepackage[left=25mm, right=25mm, top=15mm]{geometry}
\geometry{a4paper}
\usepackage{graphicx}
\usepackage{booktabs}
\usepackage{paralist}
\usepackage{subfig} 
\usepackage{fancyhdr}
\usepackage{amsmath}
\usepackage{amssymb}
\usepackage{amsfonts}
\usepackage{amsthm}
\usepackage{mathtools}
\usepackage{enumitem}
\usepackage{titlesec}
\usepackage{braket}
\usepackage{gensymb}
\usepackage{csvsimple}
\usepackage{url}
\usepackage{hyperref}
\usepackage{csquotes}
\usepackage{multicol}
\usepackage{graphicx}
\usepackage{wrapfig}
\usepackage{babel}
\usepackage{caption}
\captionsetup{font=small}
\pagestyle{fancy}
\renewcommand{\headrulewidth}{0pt}
\lhead{}\chead{}\rhead{}
\lfoot{}\cfoot{\thepage}\rfoot{}
\usepackage{sectsty}
\usepackage[nottoc,notlof,notlot]{tocbibind}
\usepackage[titles,subfigure]{tocloft}
\renewcommand{\cftsecfont}{\rmfamily\mdseries\upshape}
\renewcommand{\cftsecpagefont}{\rmfamily\mdseries\upshape}

\let\oldsection\section% Store \section
\renewcommand{\section}{% Update \section
	\renewcommand{\theequation}{\thesection.\arabic{equation}}% Update equation number
	\oldsection}% Regular \section
\let\oldsubsection\subsection% Store \subsection
\renewcommand{\subsection}{% Update \subsection
	\renewcommand{\theequation}{\thesubsection.\arabic{equation}}% Update equation number
	\oldsubsection}% Regular \subsection

\newcommand{\abs}[1]{\left\lvert#1\right\rvert}
\newcommand{\norm}[1]{\left\lVert#1\right\rVert}

\newcommand{\g}{\text{g}}
\newcommand{\m}{\text{m}}
\newcommand{\cm}{\text{cm}}
\newcommand{\mm}{\text{mm}}
\newcommand{\s}{\text{s}}
\newcommand{\N}{\text{N}}
\newcommand{\Hz}{\text{Hz}}

\newcommand{\virgolette}[1]{``\text{#1}"}
\newcommand{\tildetext}{\raise.17ex\hbox{$\scriptstyle\mathtt{\sim}$}}


\renewcommand{\arraystretch}{1.2}

\addto\captionsenglish{\renewcommand{\figurename}{Fig.}}
\addto\captionsenglish{\renewcommand{\tablename}{Tab.}}

\DeclareCaptionLabelFormat{andtable}{#1~#2  \&  \tablename~\thetable}

%opening
\title{%
    \Huge Misura indiretta della velocità della luce \\
    \Large C.d.L. in Fisica, a.a. 2022-2023 \\ Università degli Studi di Milano}
\author{\LARGE Lucrezia Bioni, Leonardo Cerasi, Giulia Federica Bianca Coppi}
\date{}

\begin{document}

    \maketitle

    \section{Introduzione}

    Lo scopo di questa esperienza è la misurazione della velocità della luce utilizzando 
    il metodo di Focault. Questa grandezza, infatti, svolge un ruolo cruciale come costante 
    fisica universale e la sua determinazione è stata di fondamentale importanza per la definizione 
    delle unità di misura nel Sistema Internazionale.

    \subsection{Metodo}
    
    La determinazione della velocità della luce viene effettuata utilizzando il metodo di Focault: 
    il fascio luminoso proveniente dalla sorgente viene diretto verso uno specchio rotante che ne causa
     la riflessione con spostamento angolare $    \Delta\omega  $.\\
     Il raggio di luce, dopo aver colpito lo specchio rotante, viene riflesso nella direzione opposta lungo 
     la stessa traiettoria che aveva compito nel viaggio di andata. Poiché lo specchio è in rotazione, 
     la posizione in cui il raggio colpisce lo specchio è in costante cambiamento: questo causa uno spostamento
      angolare tra il punto di arrivo del raggio riflesso e la posizione iniziale - misurata con specchio fermo -.\\
      Misurando con precisione la posizione iniziale $ \delta _i $ - con specchio fermo - e la finale  $ \delta _f $ - con specchio in movimento - 
      si riesce a dedurre lo spostamento angolare $ \Delta \delta $ : questo rende possibile determinare la velocità della luce 

    \begin{equation}
    \label{eqn:c}
    c=4f_2D^2
    \frac{(\omega -\omega_0)}{(D+a-f_2)\Delta \delta }
    \end{equation}

    \section{Appendice}

    \begin {table}
    \centering
    \begin{tabular}[b]{||c|c|c||}
        
        \hline
        \bfseries Colonna 1 & \bfseries Colonna 2 & \bfseries Colonna 3
        \csvreader[head to column names]{csv/CW_CCW.csv}{}
        {\\\hline\nu & \omega & \delta}

    \end{tabular}
    \caption {da frequenze massime clockwise a frequenze massime counterclockwise}
    \end {table}

    \begin {table}
    \centering
    \begin{tabular}[b]{||c|c|c||}
        
        \hline
        \bfseries Colonna 1 & \bfseries Colonna 2 & \bfseries Colonna 3
        \csvreader[head to column names]{csv/CW_min_max.csv}{}
        {\\\hline\nu & \omega & \delta}

    \end{tabular}
    \caption {clockwise da frequenze basse a frequenze massime.}
    \end {table}    
    
    \begin {table}
    \centering
    \begin{tabular}[b]{||c|c|c||}
        
        \hline
        \bfseries Colonna 1 & \bfseries Colonna 2 & \bfseries Colonna 3
        \csvreader[head to column names]{csv/CW_min_mid.csv}{}
        {\\\hline\nu & \omega & \delta}

    \end{tabular}
    \caption {clockwise da frequenze basse a frequenze medie.}
    \end {table}

\end{document}