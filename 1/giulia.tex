\documentclass[]{article}
\usepackage[utf8]{inputenc}
\usepackage[english]{babel}

\usepackage[]{csvsimple}

\usepackage{ragged2e}
\usepackage[left=25mm, right=25mm, top=15mm]{geometry}
\geometry{a4paper}
\usepackage{graphicx}
\usepackage{booktabs}
\usepackage{paralist}
\usepackage{subfig} 
\usepackage{fancyhdr}
\usepackage{amsmath}
\usepackage{amssymb}
\usepackage{amsfonts}
\usepackage{amsthm}
\usepackage{mathtools}
\usepackage{enumitem}
\usepackage{titlesec}
\usepackage{braket}
\usepackage{gensymb}
\usepackage{url}
\usepackage{hyperref}
\usepackage{csquotes}
\usepackage{multicol}
\usepackage{graphicx}
\usepackage{wrapfig}
\usepackage{babel}
\usepackage{caption}
\captionsetup{font=small}
\pagestyle{fancy}
\renewcommand{\headrulewidth}{0pt}
\lhead{}\chead{}\rhead{}
\lfoot{}\cfoot{\thepage}\rfoot{}
\usepackage{sectsty}
\usepackage[nottoc,notlof,notlot]{tocbibind}
\usepackage[titles,subfigure]{tocloft}
\renewcommand{\cftsecfont}{\rmfamily\mdseries\upshape}
\renewcommand{\cftsecpagefont}{\rmfamily\mdseries\upshape}

\let\oldsection\section% Store \section
\renewcommand{\section}{% Update \section
	\renewcommand{\theequation}{\thesection.\arabic{equation}}% Update equation number
	\oldsection}% Regular \section
\let\oldsubsection\subsection% Store \subsection
\renewcommand{\subsection}{% Update \subsection
	\renewcommand{\theequation}{\thesubsection.\arabic{equation}}% Update equation number
	\oldsubsection}% Regular \subsection

\newcommand{\abs}[1]{\left\lvert#1\right\rvert}
\newcommand{\norm}[1]{\left\lVert#1\right\rVert}

\newcommand{\g}{\text{g}}
\newcommand{\m}{\text{m}}
\newcommand{\cm}{\text{cm}}
\newcommand{\mm}{\text{mm}}
\newcommand{\s}{\text{s}}
\newcommand{\N}{\text{N}}
\newcommand{\Hz}{\text{Hz}}

\newcommand{\virgolette}[1]{``\text{#1}"}
\newcommand{\tildetext}{\raise.17ex\hbox{$\scriptstyle\mathtt{\sim}$}}


\renewcommand{\arraystretch}{1.2}

\addto\captionsenglish{\renewcommand{\figurename}{Fig.}}
\addto\captionsenglish{\renewcommand{\tablename}{Tab.}}

\DeclareCaptionLabelFormat{andtable}{#1~#2  \&  \tablename~\thetable}

%opening
\title{%
    \Huge Misura indiretta della velocità della luce \\
    \Large C.d.L. in Fisica, a.a. 2022-2023 \\ Università degli Studi di Milano}
\author{\LARGE Lucrezia Bioni, Leonardo Cerasi, Giulia Federica Bianca Coppi}
\date{}

\begin{document}

    \maketitle

    \section{Introduzione}

    Lo scopo di questa esperienza è la misurazione della velocità della luce utilizzando 
    il metodo di Focault. Questa grandezza, infatti, svolge un ruolo cruciale come costante 
    fisica universale e la sua determinazione è stata di fondamentale importanza per la definizione 
    delle unità di misura nel Sistema Internazionale.

    \subsection{Metodo}
    
    La determinazione della velocità della luce viene effettuata utilizzando il metodo di Focault: 
    viene diretto un fascio luminoso, proveniente da una sorgente, verso uno specchio rotante che ne causa
     la riflessione con spostamento angolare $    \Delta\omega  $.\\
     Il raggio di luce, dopo aver colpito lo specchio rotante, viene riflesso nella direzione opposta lungo 
     la stessa traiettoria che aveva compito nel viaggio di andata. Poiché lo specchio è in rotazione, 
     la posizione in cui il raggio colpisce lo specchio è in costante cambiamento: questo causa uno spostamento
      angolare tra il punto di arrivo del raggio riflesso e la posizione iniziale - misurata con specchio fermo -.\\
      Misurando con precisione la posizione iniziale $ \delta _i $ - con specchio fermo - e la finale  $ \delta _f $ - con specchio in movimento - 
      si riesce a dedurre lo spostamento angolare $ \Delta \delta = \delta_f - \delta_i $ : questo rende possibile determinare la velocità della luce 

    \begin{equation}
    \label{eqn:c}
    c=4f_2D^2
    \frac{(\omega -\omega_0)}{(D+a-f_2)\Delta \delta }
    \end{equation}

    dove $ c $ è la velocità della luce, $ f_2 $ è la lunghezza focale della seconda lente posta nell'apparato,
    $ D $ è la lunghezza del percorso compituto dal facio luminoso, $ \omega_0 $ e $ \omega $ sono rispettivamente
    la velocità angolare iniziale e finale dello specchio rotante, $ a $ è la distanza tra 
    la seconda lente dell'apparato e lo specchio rotante e $ \Delta\delta$ è lo spostamento 
    dell'immagine nel punto di osservazione, quando la velocità angolare dello specchio rotante passa da 
    $ \omega_0 $ a $ \omega $.

    \section {Analisi dati}

    \subsection {Stima degli errori}


    L'errore sulla stima del valore di $ c $ è frutto di due componenti: una sistematica 
    - calcolata mediante propagazione degli errori - e una statistica - data dalla deviazione standard 
    della media di $ c $ ottenuta per ogni coppia $ \Delta \omega $ e $ \Delta \delta $. \\

    $\omega$ e $\delta$ vengono considerate come grandezze prive di errore: questo risulta essere valutato nel calcolo della deviazione standard a loro riferito.\\

    La formula per la determinazione di $ c $ è quella già presentata al paragrafo 1.2 (non so come mettere la reference).
    Da essa si ricavano le seguenti formule utilizzate per la propagazione degli errori:

    \begin{equation}
        \label{eqn:propD}
        \frac{\partial c}{\partial D} = \frac{4\Delta \omega}{\Delta \delta} \frac{D f_2 (2a + D -2f_2)}{(a + D -f_2)^2}
        \end{equation}

    \begin{equation}
        \label{eqn:propa}
        \frac{\partial c}{\partial a} = -\frac{4\Delta \omega}{\Delta \delta} \frac{D^2 f_2}{(a + D -f_2)^2}
        \end{equation}


    \section{Appendice}

    \begin{table}
        \centering

    \begin{tabular}{||c|c|c||c|c|c||}
        \hline
        $\nu_0\, [\text{Hz}] $ & $\omega_0\, [\text{rad/s}] $ & $\delta_0\, [\text{m}] $ & $\nu\, [\text{Hz}] $ & $\omega_0\, [\text{rad/s}] $ & $\delta\, [\text{m}] $ \\
        \hline\hline
        $-1395$ & $-8765.043504 $ & $8.93$ & $1387$ & $8714.778021$ & $9.69$\\\hline
        $-1400$ & $-8796.459430 $ & $8.92$ & $1406$ & $8834.158542$ & $9.69$\\\hline
        $-1300$ & $-8168.140899 $ & $8.93$ & $1407$ & $8840.441727$ & $9.70$\\\hline
        $-1413$ & $-8878.140839 $ & $8.91$ & $1391$ & $8739.910762$ & $9.70$\\\hline
        $-1360$ & $-8545.132018 $ & $8.92$ & $1382$ & $8683.362095$ & $9.69$\\\hline
        $-1346$ & $-8457.167423 $ & $8.92$ & $1402$ & $8809.025801$ & $9.70$\\\hline
        $-1358$ & $-8532.565647 $ & $8.91$ & $1394$ & $8758.760318$ & $9.69$\\\hline
        $-1393$ & $-8752.477133 $ & $8.91$ & $1419$ & $8915.839951$ & $9.70$\\\hline
        $-1390$ & $-8733.627577 $ & $8.92$ & $1369$ & $8601.680686$ & $9.69$\\\hline
        $-1416$ & $-8896.990395 $ & $8.91$ & $1419$ & $8915.839951$ & $9.70$\\\hline
        $-1394$ & $-8758.760318 $ & $8.93$ & $1424$ & $8947.255877$ & $9.69$\\\hline
        $-1366$ & $-8582.831130 $ & $8.91$ & $1419$ & $8915.839951$ & $9.69$\\\hline
        $-1417$ & $-8903.273580 $ & $8.91$ & $1404$ & $8821.592171$ & $9.70$\\\hline
        $-1322$ & $-8306.370976 $ & $8.92$ & $1312$ & $8243.539123$ & $9.66$\\\hline
        $-1378$ & $-8658.229353 $ & $8.93$ & $1409$ & $8853.008098$ & $9.69$\\\hline
        $-1300$ & $-8168.140899 $ & $8.92$ & $1394$ & $8758.760318$ & $9.68$\\\hline
        $-1378$ & $-8658.229353 $ & $8.92$ & $1315$ & $8262.388679$ & $9.66$\\\hline
        $-1372$ & $-8620.530241 $ & $8.92$ & $1369$ & $8601.680686$ & $9.66$\\\hline
        $-1385$ & $-8702.211650 $ & $8.93$ & $1384$ & $8695.928465$ & $9.70$\\\hline
        $-1362$ & $-8557.698388 $ & $8.93$ & $1329$ & $8350.353273$ & $9.63$\\\hline
        $-1309$ & $-8224.689567 $ & $8.93$ & $1349$ & $8476.016979$ & $9.69$\\\hline
        $-1365$ & $-8576.547944 $ & $8.92$ & $1383$ & $8689.645280$ & $9.69$\\\hline
        $-1389$ & $-8727.344392 $ & $8.92$ & $1317$ & $8274.955050$ & $9.65$\\\hline
        $-1342$ & $-8432.034682 $ & $8.92$ & $1314$ & $8256.105494$ & $9.66$\\\hline
        $-1398$ & $-8783.893059 $ & $8.92$ & $1331$ & $8362.919644$ & $9.66$\\\hline
        $-1364$ & $-8570.264759 $ & $8.92$ & $1381$ & $8677.078909$ & $9.69$\\\hline
        $-1374$ & $-8633.096612 $ & $8.92$ & $1385$ & $8702.211650$ & $9.70$\\\hline
        $-1372$ & $-8620.530241 $ & $8.93$ & $1375$ & $8639.379797$ & $9.67$\\\hline
        $-1375$ & $-8639.379797 $ & $8.91$ & $1310$ & $8230.972752$ & $9.67$\\\hline
        $-1347$ & $-8463.450609 $ & $8.93$ & $1325$ & $8325.220532$ & $9.68$\\\hline
    \end{tabular}
    \caption{Specchio in rotazione CW a frequenza iniziale massima $\nu_0$ e in rotazione CCW a frequenza finale massima $\nu$: misure di posizione iniziale $\delta_0$ e finale $\delta$ dello spot luminoso}
    \label{CW_CCW}
\end{table}

        \begin{table}
            \centering

 \begin{tabular}{||c|c|c||c|c|c||}
        \hline
        $\nu_0\, [\text{Hz}] $ & $\omega_0\, [\text{rad/s}] $ & $\delta_0\, [\text{m}] $ & $\nu\, [\text{Hz}] $ & $\omega_0\, [\text{rad/s}] $ & $\delta\, [\text{m}] $ \\
        \hline\hline
        $14$ & $87.96459430 $ & $9.31$ & $618$ & $3883.008520$ & $9.47$\\\hline
        $17$ & $106.8141502 $ & $9.31$ & $683$ & $4291.415565$ & $9.50$\\\hline
        $18$ & $113.0973355 $ & $9.30$ & $837$ & $5259.026102$ & $9.53$\\\hline
        $13$ & $81.68140899 $ & $9.31$ & $848$ & $5328.141140$ & $9.54$\\\hline
        $14$ & $87.96459430 $ & $9.31$ & $890$ & $5592.034923$ & $9.55$\\\hline
        $20$ & $125.6637061 $ & $9.32$ & $870$ & $5466.371217$ & $9.55$\\\hline
        $11$ & $69.11503838 $ & $9.31$ & $852$ & $5353.273882$ & $9.54$\\\hline
        $18$ & $113.0973355 $ & $9.30$ & $880$ & $5529.203070$ & $9.54$\\\hline
        $14$ & $87.96459430 $ & $9.30$ & $893$ & $5610.884479$ & $9.54$\\\hline
        $14$ & $87.96459430 $ & $9.31$ & $619$ & $3889.291705$ & $9.49$\\\hline
        $13$ & $81.68140899 $ & $9.31$ & $637$ & $4002.389041$ & $9.48$\\\hline
        $15$ & $94.24777961 $ & $9.31$ & $601$ & $3776.194370$ & $9.47$\\\hline
        $13$ & $81.68140899 $ & $9.31$ & $653$ & $4102.920006$ & $9.49$\\\hline
        $15$ & $94.24777961 $ & $9.31$ & $606$ & $3807.610296$ & $9.48$\\\hline
        $17$ & $106.8141502 $ & $9.31$ & $618$ & $3883.008520$ & $9.48$\\\hline

    \end{tabular}
 
    \caption{Specchio in rotazione CCW, frequenza iniziale minima e frequenza finale intermedia: variazione di pulsazione $\Delta\omega$ e variazione di posizione $\Delta\delta$, e rispettiva misura indiretta della velocità della luce $c$}

    \label{CCW_min_mid.csv}

\end{table}

\begin{table}
    \centering

\begin{tabular}{||c|c|c||}
\hline
$\Delta\omega\, [\text{rad/s}] $ & $\Delta\delta\, [\text{m}] $ & $ $c$\, [\text{m/s}] $ \\
\hline\hline
$3795.043926$ & $0.16 $ & $3.12288E+08$ \\\hline
$4184.601415$ & $0.19 $ & $2.89974E+08$ \\\hline
$5145.928767$ & $0.23 $ & $2.94574E+08$ \\\hline
$5246.459731$ & $0.23 $ & $3.00329E+08$ \\\hline
$5504.070329$ & $0.24 $ & $3.01947E+08$ \\\hline
$5340.707511$ & $0.23 $ & $3.05724E+08$ \\\hline
$5284.158843$ & $0.23 $ & $3.02487E+08$ \\\hline
$5416.105735$ & $0.24 $ & $2.97122E+08$ \\\hline
$5522.919885$ & $0.24 $ & $3.02981E+08$ \\\hline
$3801.327111$ & $0.18 $ & $2.78049E+08$ \\\hline
$3920.707632$ & $0.17 $ & $3.03651E+08$ \\\hline
$3681.946590$ & $0.16 $ & $3.02981E+08$ \\\hline
$4021.238597$ & $0.18 $ & $2.94134E+08$ \\\hline
$3713.362517$ & $0.17 $ & $2.87592E+08$ \\\hline
$3776.194370$ & $0.17 $ & $2.92458E+08$ \\\hline

    \end{tabular}

\caption{Specchio in rotazione CCW, frequenza iniziale minima e frequenza finale intermedia: variazione di pulsazione $\Delta\omega$ e variazione di posizione $\Delta\delta$, e rispettiva misura indiretta della velocità della luce $c$}

\label{CCW_min_mid.csv}

\end{table}

\begin {table}
\centering

 \begin{tabular}{||c|c|c||c|c|c||}
    \hline
    $\nu_0\, [\text{Hz}] $ & $\omega_0\, [\text{rad/s}] $ & $\delta_0 [\text{m}] $ & $\nu\, [\text{Hz}] $ & $\omega_0\, [\text{rad/s}] $ & $\delta\, [\text{m}] $ \\
    \hline\hline
    $-14$ & $-87.96459430 $ & $9.31$ & $-684$ & $-4297.698750$ & $9.11$\\\hline
    $-16$ & $-100.5309649 $ & $9.29$ & $-811$ & $-5095.663284$ & $9.08$\\\hline
    $-20$ & $-125.6637061 $ & $9.29$ & $-739$ & $-4643.273942$ & $9.08$\\\hline
    $-17$ & $-106.8141502 $ & $9.29$ & $-860$ & $-5403.539364$ & $9.06$\\\hline
    $-19$ & $-119.3805208 $ & $9.29$ & $-867$ & $-5447.521661$ & $9.05$\\\hline
    $-17$ & $-106.8141502 $ & $9.29$ & $-841$ & $-5284.158843$ & $9.06$\\\hline
    $-16$ & $-100.5309649 $ & $9.29$ & $-865$ & $-5434.955291$ & $9.05$\\\hline
    $-15$ & $-94.24777961 $ & $9.29$ & $-856$ & $-5378.406623$ & $9.05$\\\hline
    $-19$ & $-119.3805208 $ & $9.28$ & $-839$ & $-5271.592473$ & $9.06$\\\hline
    $-15$ & $-94.24777961 $ & $9.28$ & $-892$ & $-5604.601294$ & $9.04$\\\hline
    $-13$ & $-81.68140899 $ & $9.29$ & $-840$ & $-5277.875658$ & $9.06$\\\hline
    $-23$ & $-144.5132621 $ & $9.29$ & $-844$ & $-5303.008399$ & $9.07$\\\hline
    $-13$ & $-81.68140899 $ & $9.29$ & $-874$ & $-5491.503958$ & $9.06$\\\hline
    $-21$ & $-131.9468915 $ & $9.30$ & $-889$ & $-5585.751738$ & $9.05$\\\hline
    $-16$ & $-100.5309649 $ & $9.29$ & $-885$ & $-5560.618997$ & $9.06$\\\hline

    \end{tabular}
\caption{Specchio in rotazione CW, frequenza iniziale minima e frequenza finale intermedia: variazione di pulsazione $\Delta\omega$ e variazione di posizione $\Delta\delta$, e rispettiva misura indiretta della velocità della luce $c$}

\label{CW_min_mid}
\end{table}

    %%%%%%%%%%%%%%%%%%%%%%%%%
%%
%%    \begin {table}
%%    \centering
%%    \begin{tabular}{|c|c|c|c|c|c|c|c|c|}
%%        
%%        \hline
%%        \bfseries $ \nu_0 [\text{Hz}] $ & \bfseries $ \omega_0 [\text{rad/s}] $ & \bfseries $ \delta_0 [\text{m}]$ & \bfseries $ \nu [\text{Hz}] $ & \bfseries $ \omega [\text{Hz}]$ & \bfseries $ \delta [\text{m}] $ & \bfseries Delta & \bfseries Omega & \bfseries c
%%        \csvreader[head to column names]{csv/CW_CCW.csv}{}
%%        {\\\hline\nuzero & \omegazero & \deltazero & \nu & \omega & \delta & \deltadelta & \deltaomega & \c}
%%
%%    \end{tabular}
%%    \caption {da frequenze massime clockwise a frequenze massime counterclockwise}
%%    \end {table}
%%
%%    \begin{table}
%%        \centering
%%        \begin{tabular}{|c|c|c|c|c|c|}
%%
%%            \hline
%%            \bfseries $ \nu_0 [\text{Hz}] $ & \bfseries $ \omega_0 [\text{rad/s}] $ \bfseries $ \delta_0 [\text{m}] $ 
%%
%%
%%    \begin {table}
%%    \centering
%%    \begin{tabular}[b]{||c|c|c||}
%%        
%%        \hline
%%        \bfseries $ \nu_0 $ & \bfseries Colonna 2 & \bfseries Colonna 3
%%        \csvreader[head to column names]{csv/CW_min_max.csv}{}
%%        {\\\hline\nu & \omega & \delta}
%%
%%    \end{tabular}
%%    \caption {clockwise da frequenze basse a frequenze massime.}
%%    \end {table}    
%%    
%%    \begin {table}
%%    \centering
%%    \begin{tabular}[b]{|c|c|c|c|c|c|}
%%        
%%        \hline
%%        \bfseries $ \nu_0 $ & \bfseries $ \omega_0 $ & \bfseries $ \delta_0 $ & 
%%        \bfseries $ \nu $ & \bfseries $ \omega $ & \bfseries $ \delta $
%%        \csvreader[head to column names]{csv/CW_min_mid.csv}{}
%%        {\\\hline\nu & \omega & \delta}
%%
%%    \end{tabular}
%%    \caption {clockwise da frequenze basse a frequenze medie.}
%%    \end {table}
%%
%%

\end{document}