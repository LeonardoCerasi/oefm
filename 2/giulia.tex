\documentclass[]{article}
\usepackage[utf8]{inputenc}
\usepackage[english]{babel}

\usepackage[]{csvsimple}
\usepackage{float}

\usepackage{ragged2e}
\usepackage[left=25mm, right=25mm, top=15mm]{geometry}
\geometry{a4paper}
\usepackage{graphicx}
\usepackage{booktabs}
\usepackage{paralist}
\usepackage{subfig} 
\usepackage{fancyhdr}
\usepackage{amsmath}
\usepackage{amssymb}
\usepackage{amsfonts}
\usepackage{amsthm}
\usepackage{mathtools}
\usepackage{enumitem}
\usepackage{titlesec}
\usepackage{braket}
\usepackage{gensymb}
\usepackage{url}
\usepackage{hyperref}
\usepackage{csquotes}
\usepackage{multicol}
\usepackage{graphicx}
\usepackage{wrapfig}
\usepackage{babel}
\usepackage{caption}
\captionsetup{font=small}
\pagestyle{fancy}
\renewcommand{\headrulewidth}{0pt}
\lhead{}\chead{}\rhead{}
\lfoot{}\cfoot{\thepage}\rfoot{}
\usepackage{sectsty}
\usepackage[nottoc,notlof,notlot]{tocbibind}
\usepackage[titles,subfigure]{tocloft}
\renewcommand{\cftsecfont}{\rmfamily\mdseries\upshape}
\renewcommand{\cftsecpagefont}{\rmfamily\mdseries\upshape}

\let\oldsection\section% Store \section
\renewcommand{\section}{% Update \section
	\renewcommand{\theequation}{\thesection.\arabic{equation}}% Update equation number
	\oldsection}% Regular \section
\let\oldsubsection\subsection% Store \subsection
\renewcommand{\subsection}{% Update \subsection
	\renewcommand{\theequation}{\thesubsection.\arabic{equation}}% Update equation number
	\oldsubsection}% Regular \subsection

\newcommand{\abs}[1]{\left\lvert#1\right\rvert}
\newcommand{\norm}[1]{\left\lVert#1\right\rVert}

\newcommand{\g}{\text{g}}
\newcommand{\m}{\text{m}}
\newcommand{\cm}{\text{cm}}
\newcommand{\mm}{\text{mm}}
\newcommand{\s}{\text{s}}
\newcommand{\N}{\text{N}}
\newcommand{\Hz}{\text{Hz}}

\newcommand{\virgolette}[1]{``\text{#1}"}
\newcommand{\tildetext}{\raise.17ex\hbox{$\scriptstyle\mathtt{\sim}$}}


\renewcommand{\arraystretch}{1.2}

\addto\captionsenglish{\renewcommand{\figurename}{Fig.}}
\addto\captionsenglish{\renewcommand{\tablename}{Tab.}}

\DeclareCaptionLabelFormat{andtable}{#1~#2  \&  \tablename~\thetable}


\title{%
    \Huge Misura del rapporto carica/massa di un elettrone non relativistico \\
    \Large Laboratorio di Ottica, Elettronica e Fisica Moderna \\ C.d.L. in Fisica, a.a. 2023-2024 \\ Università degli Studi di Milano}
\author{\LARGE Lucrezia Bioni, Leonardo Cerasi, Giulia Federica Bianca Coppi \\ Matricole: 13655A, 11410A, 11823A}
\date{2 novembre 2023}

\begin{document}

    \maketitle

    \section{Misure}

    \subsection{Misure preliminari}

    Innanzi tutto si prendono 6 misure - mediante l'utilizzo di un calibro di sensibilità $ 0.01\, mm $ dello spessore $ d $ del disantaziale della camera di Millikan

    \begin{table}[H]
        \centering
    
        \begin{tabular}{||c||}
            \hline
            $d\, \text{[mm]} $ \\
            \hline\hline
    
            $ 7,65 $ \\\hline
            $ 7,64 $ \\\hline
            $ 7,63 $ \\\hline
            $ 7,64 $ \\\hline
            $ 7,64 $ \\\hline
            $ 7,63 $ \\\hline

        
        \end{tabular}
        \caption{Misure dello spessore del distanziale}
        \label{distanziale}
    \end{table}

    Viene quindi attribuito come valore finale a $ d $ la sua media con la relativa incertezza strumentale:

    \begin{equation}
        \label{misura_Rb}
        d = (7.64 \pm 0.01) \, \text{mm}
    \end{equation} 

    Ogni volta che viene scelta una gocciolina da seguire, si prende la misura della temperatura attuale. Tale grandezza è variata durante l'esperimento tra i $ 21.5\, °C $ e i $ 22.0\, °C $: la temperatura relativa a ciascuna goccia è riportata nelle relative Tabb $ riferimento - alle tavelle-$.

    \subsection {Misure effettive}

    Per 10 gocce vengono prese le misure della differenza di potenziale applicata $ \Delta V $ e il tempo che questa impiega a percorrere $ 0.5 mm $. A tali grandezze si corredano ulteriori informazioni: il verso del moto - se sale o se scende - e dati qualitativi come eventuale presenza di rumore e comportamenti anomali.
    Tutti i dati vegono riporate nelle Tabb. $ riferimento alle tabelle$.

    La misurazione del tempo di volo è stata effetuata mediante un cronometro digitale di precisione $ 0.01 s $ a cui non viene attribuita alcuna incertezza e la misurazione della differenza di potenziale $ \Delta V $ è stata effettuata mediante un multimetro digitale a cui viene attribuita come incertezza $ 10 V $.


\end{document}