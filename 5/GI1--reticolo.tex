% chktex-file 1
% chktex-file 2
% chktex-file 3
% chktex-file 8
% chktex-file 9
% chktex-file 12
% chktex-file 13
% chktex-file 16
% chktex-file 18
% chktex-file 24
% chktex-file 26
% chktex-file 35
% chktex-file 44
% chktex-file 45

\documentclass[]{article}
\usepackage[utf8]{inputenc}
\usepackage[english]{babel}

\usepackage[]{csvsimple}
\usepackage{float}

\usepackage{ragged2e}
\usepackage[left=25mm, right=25mm, top=15mm]{geometry}
\geometry{a4paper}
\usepackage{graphicx}
\usepackage{booktabs}
\usepackage{paralist}
\usepackage{subfig} 
\usepackage{fancyhdr}
\usepackage{amsmath}
\usepackage{amssymb}
\usepackage{amsfonts}
\usepackage{amsthm}
\usepackage{mathtools}
\usepackage{enumitem}
\usepackage{titlesec}
\usepackage{braket}
\usepackage{gensymb}
\usepackage{url}
\usepackage{hyperref}
\usepackage{csquotes}
\usepackage{multicol}
\usepackage{graphicx}
\usepackage{wrapfig}
\usepackage{babel}
\usepackage{caption}
\captionsetup{font=small}
\pagestyle{fancy}
\renewcommand{\headrulewidth}{0pt}
\lhead{}\chead{}\rhead{}
\lfoot{}\cfoot{\thepage}\rfoot{}
\usepackage{sectsty}
\usepackage[nottoc,notlof,notlot]{tocbibind}
\usepackage[titles,subfigure]{tocloft}
\renewcommand{\cftsecfont}{\rmfamily\mdseries\upshape}
\renewcommand{\cftsecpagefont}{\rmfamily\mdseries\upshape}

\let\oldsection\section% Store \section
\renewcommand{\section}{% Update \section
	\renewcommand{\theequation}{\thesection.\arabic{equation}}% Update equation number
	\oldsection}% Regular \section
\let\oldsubsection\subsection% Store \subsection
\renewcommand{\subsection}{% Update \subsection
	\renewcommand{\theequation}{\thesubsection.\arabic{equation}}% Update equation number
	\oldsubsection}% Regular \subsection

\newcommand{\abs}[1]{\left\lvert#1\right\rvert}
\newcommand{\norm}[1]{\left\lVert#1\right\rVert}

\newcommand{\g}{\text{g}}
\newcommand{\m}{\text{m}}
\newcommand{\cm}{\text{cm}}
\newcommand{\mm}{\text{mm}}
\newcommand{\s}{\text{s}}
\newcommand{\N}{\text{N}}
\newcommand{\Hz}{\text{Hz}}

\newcommand{\virgolette}[1]{``\text{#1}"}
\newcommand{\tildetext}{\raise.17ex\hbox{$\scriptstyle\mathtt{\sim}$}}


\renewcommand{\arraystretch}{1.2}

\addto\captionsenglish{\renewcommand{\figurename}{Fig.}}
\addto\captionsenglish{\renewcommand{\tablename}{Tab.}}

\DeclareCaptionLabelFormat{andtable}{#1~#2  \&  \tablename~\thetable}

%opening
\title{%
    \Huge Misura delle lunghezze d'onda con lo spettrometro a reticolo \\
    \Large Laboratorio di Ottica, Elettronica e Fisica Moderna \\ C.d.L. in Fisica, a.a. 2023-2024 \\ Università degli Studi di Milano}
\author{\LARGE Lucrezia Bioni, Leonardo Cerasi, Giulia Federica Bianca Coppi \\ Matricole: 13655A, 11410A, 11823A}
\date{16 novembre 2023}


\begin{document}

    \maketitle

    \section{Introduzione}

    \subsection{Scopo}
    Lo scopo dell'esperienza è la misura delle lunghezze d'onda di alcune righe dello spettro di una sorgente di mercurio attraverso un reticolo in precedenza tarato con il doppietto del sodio.

    \subsection{Metodo}
    Un reticolo è un dispositivo che si presta per la misura delle lunghezze d'onda della luce ad esso incidente. Infatti, per via dei fenomeni di interferenza e diffrazione, produce un pattern caratterizzato da un'immagine centrale non dispersa e una sequenza, simmetrica rispetto al centro, di spettri. Questi sono composti da righe colorate, ciascuna corrispondente a un massimo delle varie lunghezze d'onda costituenti la luce incidente. \\
    La posizione dei vari massimi della figura di interferenza dipende dal valore della lunghezza d'onda da cui sono generati. Dunque, dopo aver determinato il passo $d$ del reticolo in uso e dopo aver misurato, ponendo il reticolo sulla piattaforma di uno spettrometro, la posizione angolare di un massimo di ordine $k=\pm 1, \pm 2,\ldots$ rispetto al massimo centrale ($\Delta \theta$), è possibile determinare la lunghezza d'onda $\lambda$ della componente del fascio incidente responsabile di quella specifica riga di spettro:
    \begin{equation}
        \label{lambda}
        \lambda = \frac{d \sin{\Delta\theta}}{k}
    \end{equation}
    Il passo $d$ del reticolo si ottiene invertendo la relazione \ref{lambda}, attraverso le misure delle posizioni angolari dei massimi di interferenza del doppietto del sodio, le cui lunghezze d'onda si assumono note:
    \begin{equation}
        \label{lambda-sodio}
        \lambda_1 = 5.890 \cdot 10^{-7} \text{m} \qquad \lambda_2 = 5.896 \cdot 10^{-7} \text{m}
    \end{equation}

    \section{Misure}
    \label{par:misure}

    \subsection{Ortogonalità}
    Per procede alle misure del passo del reticolo e delle lunghezze d'onda delle righe spettrali, è necessario preliminarmente porre il reticolo in condizione di ortogonalità rispetto al fascio di luce incidente. \\
    Per fare ciò, si misura la posizione angolare di due massimi dello stesso ordine dello spettro di una lampada a Na rispetto al massimo di diffrazione centrale, attraverso un nonio sessagesimale di risoluzione $1'$, e, se si evidenzia una differenza angolare superiore alla decina di primi di grado, si procede ad una rotazione del reticolo di un fattore correttivo $\beta$ definito da:
    \begin{equation}
        \label{beta}
        \tan{\beta} = \sin{\frac{\Delta\theta_2 - \Delta\theta_1}{2}} \frac{\cos{\frac{\Delta\theta_1 + \Delta\theta_2}{2}}}{1 - \cos{\frac{\Delta\theta_1 + \Delta\theta_2}{2}} \cos{\frac{\Delta\theta_2 - \Delta\theta_1}{2}}} \quad\Longrightarrow\quad
        \beta = \frac{\Delta\theta_2 - \Delta\theta_1}{2} \frac{\cos{\frac{\Delta\theta_1 + \Delta\theta_2}{2}}}{1 - \cos{\frac{\Delta\theta_1 + \Delta\theta_2}{2}}}
    \end{equation}
    dove $\Delta\theta_{1,2} = \theta_0 - \theta_{1,2}$ sono le posizioni angolari dei due massimi e $\theta_0 = 0.979013 \pm 0.000281 \,\text{rad}$ è la misura del massimo centrale, ottenuta come media ponderata delle misure in Tab. \ref{prel-max-c}. Poiché $\Delta\theta_1 \approx \Delta\theta_2$, è stato possibile approssimare al prim'ordine l'espressione per $\beta$. \\
    È stato necessario correggere la posizione del reticolo solo tre volte per contenere $\beta$ entro qualche primo di grado: i dati solo riportati in Tab. \ref{prel-ort}. \\
    Per non appesantire le tabelle, si è occasionalmente omesso l'errore sulla misura delle posizioni angolari assolute $\theta$: essi sono sempre pari ad $1'$.

    \subsection{Passo del reticolo}
    Per misurare il passo del reticolo, si procede all'analisi delle righe spettrali della lampada a Na, ed in particolare si è utilizzata la riga spettrale $\lambda_1 = 589.0\,\text{nm}$ all'ordine $k=4$. Le misure sono riportare in Tab. \ref{d-Na}: si è di nuovo considerato $\theta_0 = 0.979013 \pm 0.000281 \,\text{rad}$, dato dalla media ponderata dei valori in Tab. \ref{max-Na}.

    \subsection{Lunghezze d'onda del mercurio}
    Una volta determinato il passo del reticolo, si può utilizzare la relazione \ref{lambda} per determinare le lunghezze d'onda delle righe spettrali della lampada ad Hg: per fare ciò, basta effettuare varie misure delle loro posizioni angolari rispetto al massimo di diffrazione centrale, la cui posizione $\theta_0 = 0.978548 \pm 0.000092$ è calcolata come media ponderata dei valori in Tab. \ref{max-Hg}. \\
    I dati relativi a ciascuna riga spettrale sono riportati in Tabb. \ref{viola-1} - \ref{rosso}.

    \section{Analisi dati}
    \subsection{Elaborazione dati}
    \subsubsection{Passo del reticolo}
    Attraverso la relazione seguente, si è determinato il passo $d$ del reticolo: 
    \begin{equation}
        \label{passo}
        d = \frac{k \, \, \lambda_1}{\sin{\Delta\theta}}
    \end{equation}
    dove $\Delta\theta = \theta - \theta_0$.
    I valori ottenuti per ciascuna misura presa sono riportati nella Tab. \ref{d-Na}: $\sigma_d$ è stato attribuito come da Par. \ref{par:errore_passo}.
    Il valore finale di $d$ si ottiene tramite media ponderata dei valori riportati in tabella:
    \begin{equation}
        \label{d-value}
        d = 3.3766 \pm 0.0008 \,\mu\text{m}
    \end{equation}

    \subsubsection{Lunghezze d'onda del mercurio}
    Di ogni componente dello spettro del mercurio osservata, noto l'ordine e la posizione angolare del massimo considerato, si è calcolata la lunghezza d'onda $\lambda$, come da \ref{lambda}. Si è poi effettuata una media ponderata tra tutti i valori di $\lambda$ ottenuti per ciascuna componente. I valori ottenuti sono riportati nella seguente tabella:
    \begin{table}[H]
        \centering
        \begin{tabular}{||c|c|c|c||}
            \hline
            Colore & $\lambda \pm \sigma_{\lambda} \, \left[\text{nm}\right]$ & $\lambda \, \left[\text{nm}\right]$ & $\sigma$\\
            \hline \hline
            Viola 1  & $404.32 \pm 0.09$ & $404.7$ & $4.2\sigma$ \\\hline
            Indaco   & $435.57 \pm 0.09$ & $435.8$ & $2.6\sigma$ \\\hline
            Ciano    & $491.21 \pm 0.09$ & $491.6$ & $4.3\sigma$ \\\hline
            Verde    & $545.44 \pm 0.08$ & $546.0$ & $7.0\sigma$ \\\hline
            Giallo 1 & $576.46 \pm 0.08$ & $576.9$ & $5.5\sigma$ \\\hline
            Giallo 2 & $578.41 \pm 0.08$ & $579.0$ & $7.3\sigma$ \\\hline
            Rosso    & $623.09 \pm 0.10$ & $622.8$ & $2.9\sigma$ \\\hline
        \end{tabular}
        \caption{Valori di $\lambda$ di ogni componente di Hg, relativi errori, lunghezze d'onda teoriche e distanza in sigma tra valore misurato ed atteso.}
        \label{lambda-Hg-values}
    \end{table}
    dove $\sigma_{\lambda}$ è stato attribuito come da Par. \ref{par:errore_lambda}.

    \subsubsection{Potere dispersivo}
    Attraverso le misure delle posizioni angolari delle righe spettrali, si è potuto determinare il potere dispersivo $D_{m}$ del reticolo, definito come:
    \begin{equation}
        \label{D-m}
        D_m = \frac{\Delta \theta}{\Delta \lambda} \qquad \sigma_{D_m} = \sqrt{\left(\frac{\sigma_{\Delta\theta}}{\Delta\lambda}\right)^2 + \left(\frac{\Delta\theta \, \sigma_{\Delta\lambda}}{\Delta\lambda^2}\right)^2}
    \end{equation}
    dove l'incertezza è stata stimata dalla propagazione degli errori.\\
    In particolare, per sondare varie distanze angolari, sono state considerate le coppie di righe spettrali Viola 1 - Indaco, Giallo 1 - Giallo 2 e Viola 1 - Giallo 2: i relativi dati sono riportati in Tabb. \ref{viola-1-indaco}, \ref{giallo-1-giallo-2}, \ref{viola-1-giallo-2}. \\
    I valori così ottenuti vengono confrontati con quelli ricavati dalle caratteristiche del reticolo stesso tramite la relazione:
    \begin{equation}
        \label{D-t}
        D_t(\theta) = \frac{k}{d\cos{\theta}} \qquad \sigma_{D_t}(\theta) = \sqrt{\left(\frac{k \, \sigma_d}{d^2 \cos{\theta}}\right)^2 + \left(\frac{k \tan{\theta} \, \sigma_\theta}{d\cos{\theta}}\right)^2}
    \end{equation}
    dove l'incertezza è stata stimata dalla propagazione degli errori.

    \subsubsection{Potere risolutivo}
    Dalle caratteristiche del reticolo è possibile anche giustificare a posteriori il fatto che le righe spettrali Giallo 1 e 2 possano essere chiaramente distinte tra loro agli ordini $k = 2, 3, 4$; infatti, confrontando il potere risolutivo del reticolo $R(k) = k N$, dove $N$ è il numero totale di fenditure illuminate, con il rapporto $\frac{\bar{\lambda}}{\Delta\lambda}$ delle righe considerate, si può stabilire se esse appariranno separate o meno: se $\frac{\bar{\lambda}}{\Delta\lambda} < R(k)$, allora le righe spettrali appariranno separate.\\
    Assumendo che tutte le fenditure del reticolo siano investite dal fascio luminoso (lecito poiché, essendo la fenditura di fronte al collimatore dell'ordine dei decimi di mm, la sorgente può essere considerata puntiforme e sita nel fuoco del collimatore), e che quindi $N = \frac{L}{d}$, dove $L = 2 \,\text{cm}$ è la lunghezza del reticolo, si ottengono i seguenti valori:
    \begin{table}[H]
        \centering
        \begin{tabular}{||c|c|c|c|c||}
            \hline
            $k$ & $\bar{\lambda} \, \left[\text{nm}\right]$ & $\Delta\lambda \, \left[\text{nm}\right]$ & $\bar{\lambda}/\Delta\lambda$ & $R(k)$ \\
            \hline\hline
            $2$ & $577.4$ & $1.8$ & $320.8$ & $12000$ \\\hline
            $3$ & $577.3$ & $2.0$ & $289.7$ & $18000$ \\\hline
            $4$ & $577.5$ & $2.1$ & $275.0$ & $24000$ \\\hline
        \end{tabular}
        \caption{Confronto del potere risolutivo del reticolo a vari ordini.}
        \label{R}
    \end{table}

    \subsection{Stima degli errori}
    \subsection{Posizioni angolari relative al massimo centrale}
    Le singole misure di posizione relativa al massimo centrale $\Delta \theta$ sono affette da due componenti di errore: una, $\sigma_{\text{ris}}$, dovuta alla risoluzione dello strumento e una, $\sigma_{\text{ort}}$, dovuta alla non perfetta ortogonalità del reticolo rispetto alla direzione della luce incidente. L'errore complessivo, dunque, è dato dalla somma in quadratura di queste due componenti:
    \begin{equation}
        \label{sigma-delta-theta}
        \sigma_{\Delta \theta} = \sqrt{ (\sigma_{\text{ris}})^2 + (\sigma_{\text{ort}})^2 }
    \end{equation}
    Poiché $\Delta \theta$ è dato dalla differenza di due misure di posizione assoluta (quella del massimo centrale e quella del massimo di ordine k), la componente di errore dovuta alla risoluzione dello strumento è data dalla somma in quadratura degli errori sulle singole misure di posizione: $\sigma_{\text{ris}}\approx 2'$. Invece, la componente dovuta alla non perfetta ortogonalità del reticolo è stata determinata invertendo \ref{beta}, dopo aver assunto che gli scostamenti angolari causati dal $\beta$ residuo per misure effettuate a destra ($\Delta \theta_1$) e a sinistra ($\Delta \theta_2$) rispetto al massimo centrale siano uguali in modulo ma opposte in segno:
    \begin{equation}
        \label{sigma-ort}
        \sigma_{\text{ort}} = \left| \frac{(\Delta \theta_2 - \Delta \theta_1)}{2} \right| = \left| \beta \, \frac{1 - \cos{\Delta \theta}}{\cos{\Delta \theta}} \right|
    \end{equation}

    \subsubsection{Passo del reticolo}
    Per ogni misurazione effettuata, si è attribuito al valore di $d$ ottenuto un errore stimato mediante propagazione degli errori sulla grandezza $\Delta \theta$ nella formula \ref{passo}:
    \begin{equation}
        \label{d-error}
        \sigma_d = \frac{k \, \lambda \, \cos{\Delta\theta} \, \sigma_{\Delta \theta}}{\sin^2{\Delta\theta}}
    \end{equation}
    Invece, l'incertezza sul valore finale di $d$ è stata calcolata come errore della media ponderata dei valori riportati in Tab. \ref{d-Na}.
    \label{par:errore_passo}

    \subsubsection{Lunghezze d'onda del mercurio}
    Per ogni misurazione effettuata, si è attribuito al valore di $\lambda$ ottenuto un errore stimato mediante propagazione degli errori sulle grandezze $\Delta \theta$ e $d$ nella formula \ref{lambda}:
    \begin{equation}
        \label{lambda-error}
        \sigma_{\lambda} = \sqrt{\left( \frac{\sin{\Delta\theta}}{k}  \cdot \sigma_d \right)^2 + \left(\frac{d}{k} \cdot \cos{\Delta\theta} \cdot \sigma_{\Delta\theta}\right)^2  } 
    \end{equation}
    \label{par:errore_lambda}

    \section{Conclusioni}
    Il valore del passo del reticolo ottenuto $d = 3.3766 \pm 0.0008 \,\mu\text{m}$ è compatibile con la frequenza teorica delle fenditure di $\nu = 300 \,\text{mm}^{-1}$: infatti $\frac{1}{d} = 296 \,\text{mm}^{-1}$.\\
    Come si evince dai dati riportati in Tab. \ref{lambda-Hg-values}, le misure delle lunghezze d'onda non risultano essere compatibili con i valori aspettati: ciò potrebbe essere giustificabile da una sottostima dell'errore sulla misura della posizone angolare. In particolare, quest'ultima può essere viziata dalla difficoltà nell'effettivo centramento delle righe spettrali, specialmente nelle misure di ordine $k = 4$, poiché esse risultavano di maggiore larghezza angolare e minore intensità luminosa.\\
    Il potere dispersivo stimato dai dati sperimentali, nei casi di righe spettrali di lunghezze d'onda prossime tra loro (Giallo 1 - Giallo 2), risulta compatibile con il valore teorico: ciò è dovuto ad un errore percentuale molto alto. La ragione sta nella dipendenza dell'incertezza da $\frac{1}{\Delta\lambda}$. \\
    Per quanto riguarda invece le coppie di righe spettrali Viola 1 - Indaco e Viola 1 - Giallo 2, valore sperimentale e teorico non risultano compatibili: ciò può essere sempre ricollegato alla difficoltà nel centramento delle righe spettrali. \\
    Infine, i valori del potere risolutivo ottenuti giustificano la capacità di distinguere le righe spettrali Giallo 1 - Giallo 2 a tutti gli ordini considerati.

    \newpage

    \section*{Appendice}

    \begin{table}[H]
        \centering
        \begin{tabular}{||c|c||}
            \hline
            $\theta_0 \pm \sigma_{\theta_0}$ & $\theta_0 \pm \sigma_{\theta_0} \, \left[\text{rad}\right]$ \\\hline
            \hline
            $56\degree 3' \pm 1'$ & $0.978257 \pm 0.000291$ \\\hline
            $56\degree 5' \pm 1'$ & $0.978839 \pm 0.000291$ \\\hline
            $56\degree 6' \pm 1'$ & $0.979130 \pm 0.000291$ \\\hline
            $56\degree 6' \pm 1'$ & $0.979130 \pm 0.000291$ \\\hline
            $56\degree 6' \pm 1'$ & $0.979130 \pm 0.000291$ \\\hline
            $56\degree 6' \pm 1'$ & $0.979130 \pm 0.000291$ \\\hline
            $56\degree 6' \pm 1'$ & $0.979130 \pm 0.000291$ \\\hline
            $56\degree 6' \pm 1'$ & $0.979130 \pm 0.000291$ \\\hline
            $56\degree 6' \pm 1'$ & $0.979130 \pm 0.000291$ \\\hline
            $56\degree 6' \pm 1'$ & $0.979130 \pm 0.000291$ \\\hline
        \end{tabular}
        \caption{Posizione angolare del massimo di diffrazione centrale della lampada a Na prima dell'ortogonalizzazione.}
        \label{prel-max-c}
    \end{table}

    \begin{table}[H]
        \centering
        \begin{tabular}{||c|c|c|c|c||}
            \hline
            $\theta_{sx} \pm \sigma_{\theta_{sx}}$ & $\Delta\theta_{sx} \pm \sigma_{\Delta\theta_{sx}} \, \left[\text{rad}\right]$ & $\theta_{dx} \pm \sigma_{\theta_{dx}}$ & $\Delta\theta_{dx} \pm \sigma_{\Delta\theta_{dx}} \, \left[\text{rad}\right]$ & $\beta$ \\\hline
            \hline
            $99\degree  35' \pm 1'$ & $0.759044 \pm 0.000404$ & $10\degree 53' \pm 1'$ & $0.789063 \pm 0.000411$ & $130'$ \\\hline
            $99\degree  32' \pm 1'$ & $0.758171 \pm 0.000404$ & $10\degree 50' \pm 1'$ & $0.789936 \pm 0.000411$ & $137'$ \\\hline
            $100\degree 23' \pm 1'$ & $0.773006 \pm 0.000404$ & $11\degree 53' \pm 1'$ & $0.771610 \pm 0.000411$ & $ -6'$ \\\hline
        \end{tabular}
        \caption{Fattore di correzione per l'ortogonalizzazione.}
        \label{prel-ort}
    \end{table}

    \begin{table}[H]
        \centering
        \begin{tabular}{||c|c||}
            \hline
            $\theta_0 \pm \sigma_{\theta_0}$ & $\theta_0 \pm \sigma_{\theta_0} \, \left[\text{rad}\right]$ \\\hline
            \hline
            $56\degree 3' \pm 1'$ & $0.978257 \pm 0.000291$ \\\hline
            $56\degree 5' \pm 1'$ & $0.978839 \pm 0.000291$ \\\hline
            $56\degree 6' \pm 1'$ & $0.979130 \pm 0.000291$ \\\hline
            $56\degree 6' \pm 1'$ & $0.979130 \pm 0.000291$ \\\hline
            $56\degree 6' \pm 1'$ & $0.979130 \pm 0.000291$ \\\hline
            $56\degree 6' \pm 1'$ & $0.979130 \pm 0.000291$ \\\hline
            $56\degree 6' \pm 1'$ & $0.979130 \pm 0.000291$ \\\hline
            $56\degree 6' \pm 1'$ & $0.979130 \pm 0.000291$ \\\hline
            $56\degree 6' \pm 1'$ & $0.979130 \pm 0.000291$ \\\hline
            $56\degree 6' \pm 1'$ & $0.979130 \pm 0.000291$ \\\hline
        \end{tabular}
        \caption{Posizione angolare del massimo di diffrazione centrale della lampada a Na dopo l'ortogonalizzazione.}
        \label{max-Na}
    \end{table}

    \begin{table}[H]
        \centering
        \begin{tabular}{||c|c|c||}
            \hline
            $\theta \pm \sigma_{\theta}$ & $\Delta\theta_{sx} \pm \sigma_{\Delta\theta_{sx}} \, \left[\text{rad}\right]$ & $d \pm \sigma_d \,\left[\mu\text{m}\right]$ \\\hline
            \hline
            $100\degree 20' \pm 1'$ & $0.772134 \pm 0.000404$ & $3.377 \pm 0.001$ \\\hline
            $ 11\degree 54' \pm 1'$ & $0.771319 \pm 0.000404$ & $3.380 \pm 0.001$ \\\hline
            $100\degree 24' \pm 1'$ & $0.773297 \pm 0.000404$ & $3.373 \pm 0.001$ \\\hline
        \end{tabular}
        \caption{Passo del reticolo.}
        \label{d-Na}
    \end{table}

    \begin{table}[H]
        \centering
        \begin{tabular}{||c|c||}
            \hline
            $\theta_0 \pm \sigma_{\theta_0}$ & $\theta_0 \pm \sigma_{\theta_0} \, \left[\text{rad}\right]$ \\\hline
            \hline
            $56\degree 4' \pm 1'$ & $0.978548 \pm 0.000291$ \\\hline
            $56\degree 4' \pm 1'$ & $0.978548 \pm 0.000291$ \\\hline
            $56\degree 4' \pm 1'$ & $0.978548 \pm 0.000291$ \\\hline
            $56\degree 4' \pm 1'$ & $0.978548 \pm 0.000291$ \\\hline
            $56\degree 4' \pm 1'$ & $0.978548 \pm 0.000291$ \\\hline
            $56\degree 4' \pm 1'$ & $0.978548 \pm 0.000291$ \\\hline
            $56\degree 4' \pm 1'$ & $0.978548 \pm 0.000291$ \\\hline
            $56\degree 4' \pm 1'$ & $0.978548 \pm 0.000291$ \\\hline
            $56\degree 4' \pm 1'$ & $0.978548 \pm 0.000291$ \\\hline
            $56\degree 4' \pm 1'$ & $0.978548 \pm 0.000291$ \\\hline
        \end{tabular}
        \caption{Posizione angolare del massimo di diffrazione centrale della lampada ad Hg.}
        \label{max-Hg}
    \end{table}

    \begin{table}[H]
        \centering
        \begin{tabular}{||c|c|c|c||}
            \hline
            $k$ & $\theta \pm \sigma_{\theta}$ & $\Delta\theta \pm \sigma_{\Delta\theta} \, \left[\text{rad}\right]$ & $\lambda \pm \sigma_{\lambda} \, \left[\text{nm}\right]$ \\\hline
            \hline
            $2$ & $42\degree 13' \pm 1'$ & $0.241728 \pm 0.000305$ & $404.1 \pm 0.5$ \\\hline
            $3$ & $35\degree  0' \pm 1'$ & $0.367683 \pm 0.000305$ & $404.5 \pm 0.3$ \\\hline
            $4$ & $27\degree 25' \pm 1'$ & $0.500037 \pm 0.000305$ & $404.7 \pm 0.2$ \\\hline
            $4$ & $27\degree 28' \pm 1'$ & $0.499164 \pm 0.000305$ & $404.1 \pm 0.2$ \\\hline
            $4$ & $27\degree 27' \pm 1'$ & $0.499455 \pm 0.000305$ & $404.3 \pm 0.2$ \\\hline
            $4$ & $27\degree 28' \pm 1'$ & $0.499164 \pm 0.000305$ & $404.1 \pm 0.2$ \\\hline
            $4$ & $27\degree 28' \pm 1'$ & $0.499164 \pm 0.000305$ & $404.1 \pm 0.2$ \\\hline
            $3$ & $35\degree  0' \pm 1'$ & $0.367683 \pm 0.000305$ & $404.6 \pm 0.3$ \\\hline
            $3$ & $35\degree  0' \pm 1'$ & $0.367683 \pm 0.000305$ & $404.6 \pm 0.3$ \\\hline
            $3$ & $35\degree  1' \pm 1'$ & $0.367392 \pm 0.000305$ & $404.3 \pm 0.3$ \\\hline
            $3$ & $35\degree  1' \pm 1'$ & $0.367392 \pm 0.000305$ & $404.3 \pm 0.3$ \\\hline
        \end{tabular}
        \caption{Analisi della riga spettrale Viola 1 a vari ordini.}
        \label{viola-1}
    \end{table}

    \begin{table}[H]
        \centering
        \begin{tabular}{||c|c|c|c||}
            \hline
            $k$ & $\theta \pm \sigma_{\theta}$ & $\Delta\theta \pm \sigma_{\Delta\theta} \, \left[\text{rad}\right]$ & $\lambda \pm \sigma_{\lambda} \, \left[\text{nm}\right]$ \\\hline
            \hline
            $2$ & $41\degree  8' \pm 1'$ & $0.260636 \pm 0.000305$ & $435.1 \pm 0.5$ \\\hline
            $3$ & $33\degree 18' \pm 1'$ & $0.397353 \pm 0.000305$ & $435.6 \pm 0.3$ \\\hline
            $4$ & $25\degree  0' \pm 1'$ & $0.542216 \pm 0.000305$ & $435.6 \pm 0.2$ \\\hline
            $4$ & $25\degree  0' \pm 1'$ & $0.542216 \pm 0.000305$ & $435.6 \pm 0.2$ \\\hline
            $4$ & $25\degree  0' \pm 1'$ & $0.542216 \pm 0.000305$ & $435.6 \pm 0.2$ \\\hline
            $4$ & $25\degree  0' \pm 1'$ & $0.542216 \pm 0.000305$ & $435.6 \pm 0.2$ \\\hline
            $4$ & $25\degree  1' \pm 1'$ & $0.541925 \pm 0.000305$ & $435.4 \pm 0.2$ \\\hline
            $3$ & $33\degree 18' \pm 1'$ & $0.397353 \pm 0.000305$ & $435.6 \pm 0.3$ \\\hline
            $3$ & $33\degree 17' \pm 1'$ & $0.397644 \pm 0.000305$ & $435.6 \pm 0.3$ \\\hline
            $3$ & $33\degree 18' \pm 1'$ & $0.397353 \pm 0.000305$ & $435.6 \pm 0.3$ \\\hline
            $3$ & $33\degree 18' \pm 1'$ & $0.397353 \pm 0.000305$ & $435.6 \pm 0.3$ \\\hline
        \end{tabular}
        \caption{Analisi della riga spettrale Indaco a vari ordini.}
        \label{indaco}
    \end{table}
    
    \begin{table}[H]
        \centering
        \begin{tabular}{||c|c|c|c||}
            \hline
            $k$ & $\theta \pm \sigma_{\theta}$ & $\Delta\theta \pm \sigma_{\Delta\theta} \, \left[\text{rad}\right]$ & $\lambda \pm \sigma_{\lambda} \, \left[\text{nm}\right]$ \\\hline
            \hline
            $2$ & $39\degree 10' \pm 1'$ & $0.294961 \pm 0.000305$ & $491.0 \pm 0.5$ \\\hline
            $3$ & $30\degree 11' \pm 1'$ & $0.451749 \pm 0.000305$ & $491.3 \pm 0.3$ \\\hline
            $4$ & $20\degree 30' \pm 1'$ & $0.620755 \pm 0.000305$ & $491.0 \pm 0.2$ \\\hline
            $4$ & $20\degree 28' \pm 1'$ & $0.621337 \pm 0.000305$ & $491.4 \pm 0.2$ \\\hline
            $4$ & $20\degree 29' \pm 1'$ & $0.621046 \pm 0.000305$ & $491.2 \pm 0.2$ \\\hline
            $4$ & $20\degree 30' \pm 1'$ & $0.620755 \pm 0.000305$ & $491.0 \pm 0.2$ \\\hline
            $4$ & $20\degree 28' \pm 1'$ & $0.621337 \pm 0.000305$ & $491.4 \pm 0.2$ \\\hline
            $3$ & $30\degree 12' \pm 1'$ & $0.451458 \pm 0.000305$ & $491.0 \pm 0.3$ \\\hline
            $3$ & $30\degree 11' \pm 1'$ & $0.451749 \pm 0.000305$ & $491.3 \pm 0.3$ \\\hline
            $3$ & $30\degree 11' \pm 1'$ & $0.451749 \pm 0.000305$ & $491.3 \pm 0.3$ \\\hline
            $3$ & $30\degree 11' \pm 1'$ & $0.451749 \pm 0.000305$ & $491.3 \pm 0.3$ \\\hline
        \end{tabular}
        \caption{Analisi della riga spettrale Ciano a vari ordini.}
        \label{ciano}
    \end{table}

    \begin{table}[H]
        \centering
        \begin{tabular}{||c|c|c|c||}
            \hline
            $k$ & $\theta \pm \sigma_{\theta}$ & $\Delta\theta \pm \sigma_{\Delta\theta} \, \left[\text{rad}\right]$ & $\lambda \pm \sigma_{\lambda} \, \left[\text{nm}\right]$ \\\hline
            \hline
            $2$ & $37\degree 13' \pm 1'$ & $0.328995 \pm 0.000305$ & $545.5 \pm 0.5$ \\\hline
            $3$ & $27\degree  5' \pm 1'$ & $0.505855 \pm 0.000305$ & $545.4 \pm 0.3$ \\\hline
            $4$ & $15\degree 49' \pm 1'$ & $0.702495 \pm 0.000305$ & $545.4 \pm 0.2$ \\\hline
            $4$ & $15\degree 48' \pm 1'$ & $0.702786 \pm 0.000305$ & $445.6 \pm 0.2$ \\\hline
            $4$ & $15\degree 49' \pm 1'$ & $0.702495 \pm 0.000305$ & $445.4 \pm 0.2$ \\\hline
            $4$ & $15\degree 49' \pm 1'$ & $0.702495 \pm 0.000305$ & $445.4 \pm 0.2$ \\\hline
            $4$ & $15\degree 48' \pm 1'$ & $0.702786 \pm 0.000305$ & $445.6 \pm 0.2$ \\\hline
            $3$ & $27\degree  5' \pm 1'$ & $0.505855 \pm 0.000305$ & $491.0 \pm 0.3$ \\\hline
            $3$ & $27\degree  5' \pm 1'$ & $0.505855 \pm 0.000305$ & $491.3 \pm 0.3$ \\\hline
            $3$ & $27\degree  6' \pm 1'$ & $0.505564 \pm 0.000305$ & $491.3 \pm 0.3$ \\\hline
            $3$ & $27\degree  5' \pm 1'$ & $0.505855 \pm 0.000305$ & $491.3 \pm 0.3$ \\\hline
        \end{tabular}
        \caption{Analisi della riga spettrale Verde a vari ordini.}
        \label{verde}
    \end{table}

    \begin{table}[H]
        \centering
        \begin{tabular}{||c|c|c|c||}
            \hline
            $k$ & $\theta \pm \sigma_{\theta}$ & $\Delta\theta \pm \sigma_{\Delta\theta} \, \left[\text{rad}\right]$ & $\lambda \pm \sigma_{\lambda} \, \left[\text{nm}\right]$ \\\hline
            \hline
            $2$ & $36\degree  6' \pm 1'$ & $0.348484 \pm 0.000305$ & $576.5 \pm 0.5$ \\\hline
            $3$ & $25\degree 16' \pm 1'$ & $0.537561 \pm 0.000305$ & $576.3 \pm 0.3$ \\\hline
            $4$ & $13\degree  0' \pm 1'$ & $0.751655 \pm 0.000305$ & $576.4 \pm 0.2$ \\\hline
            $4$ & $13\degree  0' \pm 1'$ & $0.751655 \pm 0.000305$ & $576.2 \pm 0.2$ \\\hline
            $4$ & $13\degree  0' \pm 1'$ & $0.751655 \pm 0.000305$ & $576.2 \pm 0.2$ \\\hline
            $4$ & $13\degree  1' \pm 1'$ & $0.751364 \pm 0.000305$ & $576.4 \pm 0.2$ \\\hline
            $4$ & $13\degree  0' \pm 1'$ & $0.751655 \pm 0.000305$ & $576.2 \pm 0.2$ \\\hline
            $3$ & $25\degree 14' \pm 1'$ & $0.538143 \pm 0.000305$ & $576.9 \pm 0.3$ \\\hline
            $3$ & $25\degree 16' \pm 1'$ & $0.537561 \pm 0.000305$ & $576.3 \pm 0.3$ \\\hline
            $3$ & $25\degree 14' \pm 1'$ & $0.538143 \pm 0.000305$ & $576.9 \pm 0.3$ \\\hline
            $3$ & $25\degree 15' \pm 1'$ & $0.537852 \pm 0.000305$ & $576.6 \pm 0.3$ \\\hline
        \end{tabular}
        \caption{Analisi della riga spettrale Giallo 1 a vari ordini.}
        \label{giallo-1}
    \end{table}

    \begin{table}[H]
        \centering
        \begin{tabular}{||c|c|c|c||}
            \hline
            $k$ & $\theta \pm \sigma_{\theta}$ & $\Delta\theta \pm \sigma_{\Delta\theta} \, \left[\text{rad}\right]$ & $\lambda \pm \sigma_{\lambda} \, \left[\text{nm}\right]$ \\\hline
            \hline
            $2$ & $36\degree  2' \pm 1'$ & $0.349648 \pm 0.000305$ & $578.3 \pm 0.5$ \\\hline
            $3$ & $25\degree  9' \pm 1'$ & $0.539598 \pm 0.000305$ & $578.3 \pm 0.3$ \\\hline
            $4$ & $12\degree 48' \pm 1'$ & $0.755146 \pm 0.000305$ & $578.6 \pm 0.2$ \\\hline
            $4$ & $12\degree 48' \pm 1'$ & $0.755146 \pm 0.000305$ & $578.6 \pm 0.2$ \\\hline
            $4$ & $12\degree 48' \pm 1'$ & $0.755146 \pm 0.000305$ & $578.6 \pm 0.2$ \\\hline
            $4$ & $12\degree 48' \pm 1'$ & $0.755146 \pm 0.000305$ & $578.6 \pm 0.2$ \\\hline
            $4$ & $12\degree 49' \pm 1'$ & $0.754855 \pm 0.000305$ & $578.4 \pm 0.2$ \\\hline
            $3$ & $25\degree 10' \pm 1'$ & $0.539307 \pm 0.000305$ & $578.0 \pm 0.3$ \\\hline
            $3$ & $25\degree  9' \pm 1'$ & $0.539598 \pm 0.000305$ & $578.3 \pm 0.3$ \\\hline
            $3$ & $25\degree  9' \pm 1'$ & $0.539598 \pm 0.000305$ & $578.3 \pm 0.3$ \\\hline
            $3$ & $25\degree 10' \pm 1'$ & $0.539307 \pm 0.000305$ & $578.0 \pm 0.3$ \\\hline
        \end{tabular}
        \caption{Analisi della riga spettrale Giallo 2 a vari ordini.}
        \label{giallo-2}
    \end{table}

    \begin{table}[H]
        \centering
        \begin{tabular}{||c|c|c|c|c|c||}
            \hline
            $k$ & $\theta \pm \sigma_{\theta}$ & $\Delta\theta \pm \sigma_{\Delta\theta} \, \left[\text{rad}\right]$ & $\lambda \pm \sigma_{\lambda} \, \left[\text{nm}\right]$ \\\hline
            \hline
            $2$ & $34\degree 25' \pm 1'$ & $0.377864 \pm 0.000305$ & $622.9 \pm 0.5$ \\\hline
            $3$ & $22\degree 28' \pm 1'$ & $0.586431 \pm 0.000305$ & $622.9 \pm 0.3$ \\\hline
            $3$ & $22\degree 28' \pm 1'$ & $0.586431 \pm 0.000305$ & $622.9 \pm 0.2$ \\\hline
            $2$ & $22\degree 26' \pm 1'$ & $0.587012 \pm 0.000305$ & $622.3 \pm 0.2$ \\\hline
            $3$ & $22\degree 28' \pm 1'$ & $0.586431 \pm 0.000305$ & $622.3 \pm 0.2$ \\\hline
            $3$ & $22\degree 27' \pm 1'$ & $0.586722 \pm 0.000305$ & $622.3 \pm 0.2$ \\\hline
            $3$ & $22\degree 27' \pm 1'$ & $0.586722 \pm 0.000305$ & $622.3 \pm 0.2$ \\\hline
            $3$ & $22\degree 27' \pm 1'$ & $0.367683 \pm 0.000305$ & $623.1 \pm 0.3$ \\\hline
            $3$ & $22\degree 27' \pm 1'$ & $0.367683 \pm 0.000305$ & $623.1 \pm 0.3$ \\\hline
            $3$ & $22\degree 26' \pm 1'$ & $0.367392 \pm 0.000305$ & $623.4 \pm 0.3$ \\\hline
            $3$ & $22\degree 27' \pm 1'$ & $0.367392 \pm 0.000305$ & $623.1 \pm 0.3$ \\\hline
        \end{tabular}
        \caption{Analisi della riga spettrale Rosso a vari ordini.}
        \label{rosso}
    \end{table}

    \begin{table}[H]
        \centering
        \begin{tabular}{||c|c|c|c|c||}
            \hline
            $k$ & $\theta_1 \pm \sigma_{\theta_1} \,\left[\text{rad}\right]$ & $\lambda_1 \pm \sigma_ {\lambda_1} \,\left[\text{nm}\right]$ & $\theta_2 \pm \sigma_{\theta_2} \,\left[\text{rad}\right]$ & $\lambda_2 \pm \sigma_ {\lambda_2} \,\left[\text{nm}\right]$ \\\hline
            \hline
            $2$ & $0.736820 \pm 0.000291$ & $404.1 \pm 0.5$ & $0.717912 \pm 0.000291$ & $435.1 \pm 0.5$ \\\hline
            $3$ & $0.610865 \pm 0.000291$ & $404.6 \pm 0.3$ & $0.581195 \pm 0.000291$ & $435.6 \pm 0.3$ \\\hline
            $4$ & $0.478511 \pm 0.000291$ & $404.7 \pm 0.2$ & $0.436332 \pm 0.000291$ & $435.6 \pm 0.2$ \\\hline
            \hline
            $k$ & $\Delta\theta \pm \sigma_{\Delta\theta} \,\left[\text{rad}\right]$ & $\Delta\lambda \pm \sigma_{\Delta\lambda} \,\left[\text{nm}\right]$ & $D_m \pm \sigma_{D_m} \,\left[\cdot 10^5 \text{rad/m}\right]$ & $D_t \pm \sigma_{D_t} \,\left[\cdot 10^5 \text{rad/m}\right]$ \\\hline
            \hline
            $2$ & $0.018908 \pm 0.000411$ & $30.9 \pm 0.7$ & $6.1 \pm 0.2$ & $7.930 \pm 0.003$ \\\hline
            $3$ & $0.029671 \pm 0.000411$ & $31.0 \pm 0.5$ & $9.6 \pm 0.2$ & $10.740 \pm 0.003$ \\\hline
            $4$ & $0.042179 \pm 0.000411$ & $30.9 \pm 0.3$ & $14.0\pm 0.2$ & $13.200\pm 0.004$ \\\hline
        \end{tabular}
        \caption{Potere dispersivo relativo alle righe spettrali Viola 1 e Indaco.}
        \label{viola-1-indaco}
    \end{table}

    \begin{table}[H]
        \centering
        \begin{tabular}{||c|c|c|c|c||}
            \hline
            $k$ & $\theta_1 \pm \sigma_{\theta_1} \,\left[\text{rad}\right]$ & $\lambda_1 \pm \sigma_ {\lambda_1} \,\left[\text{nm}\right]$ & $\theta_2 \pm \sigma_{\theta_2} \,\left[\text{rad}\right]$ & $\lambda_2 \pm \sigma_ {\lambda_2} \,\left[\text{nm}\right]$ \\\hline
            \hline
            $2$ & $0.630064 \pm 0.000291$ & $576.5 \pm 0.5$ & $0.628900 \pm 0.000291$ & $578.3 \pm 0.5$ \\\hline
            $3$ & $0.440987 \pm 0.000291$ & $576.3 \pm 0.3$ & $0.438950 \pm 0.000291$ & $578.3 \pm 0.3$ \\\hline
            $4$ & $0.226893 \pm 0.000291$ & $576.4 \pm 0.2$ & $0.223402 \pm 0.000291$ & $578.6 \pm 0.2$ \\\hline
            \hline
            $k$ & $\Delta\theta \pm \sigma_{\Delta\theta} \,\left[\text{rad}\right]$ & $\Delta\lambda \pm \sigma_{\Delta\lambda} \,\left[\text{nm}\right]$ & $D_m \pm \sigma_{D_m} \,\left[\cdot 10^5 \text{rad/m}\right]$ & $D_t \pm \sigma_{D_t} \,\left[\cdot 10^5 \text{rad/m}\right]$ \\\hline
            \hline
            $2$ & $0.001164 \pm 0.000411$ & $1.8 \pm 0.7$ & $6 \pm 3$ & $7.328 \pm 0.002$ \\\hline
            $3$ & $0.002036 \pm 0.000411$ & $2.0 \pm 0.4$ & $10\pm 3$ & $9.820 \pm 0.003$ \\\hline
            $4$ & $0.003491 \pm 0.000411$ & $2.1 \pm 0.3$ & $16\pm 3$ & $12.150\pm 0.003$ \\\hline
        \end{tabular}
        \caption{Potere dispersivo relativo alle righe spettrali Giallo 1 e 2.}
        \label{giallo-1-giallo-2}
    \end{table}

    \begin{table}[H]
        \centering
        \begin{tabular}{||c|c|c|c|c||}
            \hline
            $k$ & $\theta_1 \pm \sigma_{\theta_1} \,\left[\text{rad}\right]$ & $\lambda_1 \pm \sigma_ {\lambda_1} \,\left[\text{nm}\right]$ & $\theta_2 \pm \sigma_{\theta_2} \,\left[\text{rad}\right]$ & $\lambda_2 \pm \sigma_ {\lambda_2} \,\left[\text{nm}\right]$ \\\hline
            \hline
            $2$ & $0.736820 \pm 0.000291$ & $404.1 \pm 0.5$ & $0.630064 \pm 0.000291$ & $576.5 \pm 0.5$ \\\hline
            $3$ & $0.610865 \pm 0.000291$ & $404.6 \pm 0.3$ & $0.440987 \pm 0.000291$ & $576.3 \pm 0.3$ \\\hline
            $4$ & $0.478511 \pm 0.000291$ & $404.7 \pm 0.2$ & $0.226893 \pm 0.000291$ & $576.4 \pm 0.2$ \\\hline
            \hline
            $k$ & $\Delta\theta \pm \sigma_{\Delta\theta} \,\left[\text{rad}\right]$ & $\Delta\lambda \pm \sigma_{\Delta\lambda} \,\left[\text{nm}\right]$ & $D_m \pm \sigma_{D_m} \,\left[\cdot 10^5 \text{rad/m}\right]$ & $D_t \pm \sigma_{D_t} \,\left[\cdot 10^5 \text{rad/m}\right]$ \\\hline
            \hline
            $2$ & $0.106756 \pm 0.000411$ & $172.4 \pm 0.7$ & $6.19 \pm 0.03$ & $7.639 \pm 0.003$ \\\hline
            $3$ & $0.169879 \pm 0.000411$ & $171.7 \pm 0.4$ & $9.89 \pm 0.04$ & $10.270\pm 0.003$ \\\hline
            $4$ & $0.251618 \pm 0.000411$ & $171.7 \pm 0.3$ & $14.70\pm 0.04$ & $12.620\pm 0.003$ \\\hline
        \end{tabular}
        \caption{Potere dispersivo relativo alle righe spettrali Viola 1 e Giallo 2.}
        \label{viola-1-giallo-2}
    \end{table}

\end{document}