\documentclass[]{article}
\usepackage[utf8]{inputenc}
\usepackage[english]{babel}

\usepackage[]{csvsimple}
\usepackage{float}

\usepackage{ragged2e}
\usepackage[left=25mm, right=25mm, top=15mm]{geometry}
\geometry{a4paper}
\usepackage{graphicx}
\usepackage{booktabs}
\usepackage{paralist}
\usepackage{subfig} 
\usepackage{fancyhdr}
\usepackage{amsmath}
\usepackage{amssymb}
\usepackage{amsfonts}
\usepackage{amsthm}
\usepackage{mathtools}
\usepackage{enumitem}
\usepackage{titlesec}
\usepackage{braket}
\usepackage{gensymb}
\usepackage{url}
\usepackage{hyperref}
\usepackage{csquotes}
\usepackage{multicol}
\usepackage{graphicx}
\usepackage{wrapfig}
\usepackage{babel}
\usepackage{caption}
\captionsetup{font=small}
\pagestyle{fancy}
\renewcommand{\headrulewidth}{0pt}
\lhead{}\chead{}\rhead{}
\lfoot{}\cfoot{\thepage}\rfoot{}
\usepackage{sectsty}
\usepackage[nottoc,notlof,notlot]{tocbibind}
\usepackage[titles,subfigure]{tocloft}
\renewcommand{\cftsecfont}{\rmfamily\mdseries\upshape}
\renewcommand{\cftsecpagefont}{\rmfamily\mdseries\upshape}

\let\oldsection\section% Store \section
\renewcommand{\section}{% Update \section
	\renewcommand{\theequation}{\thesection.\arabic{equation}}% Update equation number
	\oldsection}% Regular \section
\let\oldsubsection\subsection% Store \subsection
\renewcommand{\subsection}{% Update \subsection
	\renewcommand{\theequation}{\thesubsection.\arabic{equation}}% Update equation number
	\oldsubsection}% Regular \subsection

\newcommand{\abs}[1]{\left\lvert#1\right\rvert}
\newcommand{\norm}[1]{\left\lVert#1\right\rVert}

\newcommand{\g}{\text{g}}
\newcommand{\m}{\text{m}}
\newcommand{\cm}{\text{cm}}
\newcommand{\mm}{\text{mm}}
\newcommand{\s}{\text{s}}
\newcommand{\N}{\text{N}}
\newcommand{\Hz}{\text{Hz}}

\newcommand{\virgolette}[1]{``\text{#1}"}
\newcommand{\tildetext}{\raise.17ex\hbox{$\scriptstyle\mathtt{\sim}$}}


\renewcommand{\arraystretch}{1.2}

\addto\captionsenglish{\renewcommand{\figurename}{Fig.}}
\addto\captionsenglish{\renewcommand{\tablename}{Tab.}}

\DeclareCaptionLabelFormat{andtable}{#1~#2  \&  \tablename~\thetable}

%opening
\title{%
    \Huge Misura delle lunghezze d'onda con lo spettrometro a reticolo \\
    \Large Laboratorio di Ottica, Elettronica e Fisica Moderna \\ C.d.L. in Fisica, a.a. 2023-2024 \\ Università degli Studi di Milano}
\author{\LARGE Lucrezia Bioni, Leonardo Cerasi, Giulia Federica Bianca Coppi \\ Matricole: 13655A, 11410A, 11823A}
\date{16 novembre 2023}


\begin{document}

    \maketitle

    \section{Introduzione}

    \subsection{Scopo}
    Lo scopo dell'esperienza è la misura delle lunghezze d'onda di alcune righe dello spettro di una sorgente di mercurio attraverso un reticolo in precedenza tarato con il doppietto del sodio.

    \subsection{Metodo}
    Un reticolo è un dispositivo che si presta per la misura delle lunghezze d'onda della luce ad esso incidente. Infatti, per via dei fenomeni di interferenza e diffrazione, produce un pattern caratterizzato da un'immagine centrale non dispersa e una sequenza, simmetrica rispetto al centro, di spettri. Questi sono composti da righe colorate, ciascuna corrispondente a un massimo delle varie lunghezze d'onda costituenti la luce incidente. \\
    La posizione dei vari massimi della figura di interferenza dipende dal valore della lunghezza d'onda da cui sono generati. Dunque, dopo aver determinato il passo $d$ del reticolo in uso e dopo aver misurato, ponendo il reticolo sulla piattaforma di uno spettrometro, la posizione angolare di un massimo di ordine $k=\pm 1, \pm 2,\ldots$ rispetto al massimo centrale ($\Delta \theta$), è possibile determinare la lunghezza d'onda $\lambda$ della componente del fascio incidente responsabile di quella specifica riga di spettro:
    \begin{equation}
        \label{lambda}
        \lambda = \frac{d \, \, \sin{\Delta\theta}}{k}
    \end{equation}
    Il passo $d$ del reticolo si ottiene invertendo la relazione \ref{lambda}, attraverso le misure delle posizioni angolari dei massimi di interferenza del doppietto del sodio, le cui lunghezze d'onda si assumono note:
    \begin{equation}
        \label{lambda-sodio-1}
        \lambda_1 = 5.890 \cdot 10^{-7} \text{m}
    \end{equation}
    \begin{equation}
        \label{lambda-sodio-2}
        \lambda_2 = 5.896 \cdot 10^{-7} \text{m}
    \end{equation}

    \section{Misure}

    \section{Analisi dati}
    \subsection{Elaborazione dati}
    \subsubsection{Passo del reticolo}
    Attraverso la relazione seguente, si è determinato il passo $d$ del reticolo: 
    \begin{equation}
        \label{passo}
        d = \frac{k \, \, \lambda_1}{\sin{\Delta\theta}}
    \end{equation}
    Dove $\Delta\theta$ è la posizione angolare del massimo di ordine $k=4$ della lunghezza d'onda $\lambda_1$ del Na.
    I valori ottenuti per ciascuna misura presa sono riportati nella seguente tabella:
    \begin{table} [H]
        \centering
        \begin{tabular}{||c|c|c||}
            \hline
            N° misura & $d [\cdot 10^{-6}\text{ m }] $ & $ \sigma_d [\cdot 10^{-6}\text{ m}] $\\
            \hline \hline
            $ 1 $ & $ 3.3770 $ & $ 0.0014 $ \\\hline
            $ 2 $ & $ 3.3798 $ & $ 0.0014 $ \\\hline
            $ 3 $ & $ 3.3729 $ & $ 0.0014 $ \\\hline
        \end{tabular}
        \caption{Valori di $d$ e relativi errori.}
        \label{d-values}
    \end{table}
    Dove $\sigma_d$ è stato attribuito come da Par. \ref{par:errore_passo}.
    Il valore finale di $d$ si ottiene mediante media pesata dei valori riportati in tabella:
    \begin{equation}
        \label{d-value}
        d = (3.3766 \pm 0.0008 ) \cdot 10^{-6} \text{m}
    \end{equation}

    \subsubsection{Lunghezze d'onda di Hg}
    Di ogni componente dello spettro del mercurio osservata, noto l'ordine e la posizione angolare del massimo considerato, si è calcolata la lunghezza d'onda $\lambda$, come da \ref{lambda}. Si è poi effettuata una media ponderata tra tutti i valori di $\lambda$ ottenuti per ciascuna componente. I valori ottenuti sono riportati nella seguente tabella:
    \begin{table} [H]
        \centering
        \begin{tabular}{||c|c|c||}
            \hline
            Colore & $\lambda [\cdot 10^{-7}\text{ m }] $ & $ \sigma_{\lambda} [\cdot 10^{-7}\text{ m}] $\\
            \hline \hline
            Viola & $ 4.0432 $ & $ 0.0008 $ \\\hline
            Indaco & $ 4.3557 $ & $ 0.0008 $ \\\hline
            Ciano & $ 4.9121 $ & $ 0.0008 $ \\\hline
            Verde & $ 5.4544 $ & $ 0.0008 $ \\\hline
            Giallo 1 & $ 5.7646 $ & $ 0.0008 $ \\\hline
            Giallo 2 & $ 5.7841 $ & $ 0.0008 $ \\\hline
            Rosso & $ 6.2309 $ & $ 0.0010 $ \\\hline
        \end{tabular}
        \caption{Valori di $\lambda$ di ogni componente di Hg e relativi errori.}
        \label{lambda-Hg-values}
    \end{table}
    Dove $\sigma_{\lambda}$ è stato attribuito come da Par. \ref{par:errore_lambda}.

    \subsubsection{Potere dispersivo}
    Attraverso le misure delle posizioni angolari di righe spettrali, si è potuto determinare il potere dispersivo $D_{m}$ del reticolo, definito come:
    \begin{equation}
        D_m = \frac{\Delta \theta}{\Delta \lambda}
    \end{equation}

    \subsubsection{Potere risolutivo}
    


    \subsection{Stima degli errori}
    \subsubsection{Passo del reticolo}
    Per ogni misurazione effettuata, si è attribuito al valore di $d$ ottenuto un errore stimato mediante propagazione degli errori sulla grandezza $\Delta \theta$ nella formula \ref{passo}:
    \begin{equation}
        \label{d-error}
        \sigma_d = \frac{k \, \lambda \, \cos{\Delta\theta} \, \sigma_{\Delta \theta}}{\sin^2{\Delta\theta}}
    \end{equation}
    Invece, l'incertezza sul valore finale di $d$ è stata calcolata come errore di una media ponderata dei valori riportati in Tab. \ref{d-values}.
    \label{par:errore_passo}

    \subsubsection{Lunghezze d'onda di Hg}
    Per ogni misurazione effettuata, si è attribuito al valore di $\lambda$ ottenuto un errore stimato mediante propagazione degli errori sulle grandezze $\Delta \theta$ e $d$ nella formula \ref{lambda}:
    \begin{equation}
        \label{lambda-error}
        \sigma_{\lambda} = \sqrt{\left( \frac{\sin{\Delta\theta}}{k}  \cdot \sigma_d \right)^2 + \left(\frac{d}{k} \cdot \cos{\Delta\theta} \cdot \sigma_{\Delta\theta}\right)^2  } 
    \end{equation}
    \label{par:errore_lambda}

    \section{Conclusioni}

\end{document}