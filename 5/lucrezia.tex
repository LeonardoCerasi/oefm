\documentclass[]{article}
\usepackage[utf8]{inputenc}
\usepackage[english]{babel}

\usepackage[]{csvsimple}

\usepackage{ragged2e}
\usepackage[left=25mm, right=25mm, top=15mm]{geometry}
\geometry{a4paper}
\usepackage{graphicx}
\usepackage{booktabs}
\usepackage{paralist}
\usepackage{subfig} 
\usepackage{fancyhdr}
\usepackage{amsmath}
\usepackage{amssymb}
\usepackage{amsfonts}
\usepackage{amsthm}
\usepackage{mathtools}
\usepackage{enumitem}
\usepackage{titlesec}
\usepackage{braket}
\usepackage{gensymb}
\usepackage{url}
\usepackage{hyperref}
\usepackage{csquotes}
\usepackage{multicol}
\usepackage{graphicx}
\usepackage{wrapfig}
\usepackage{babel}
\usepackage{caption}
\captionsetup{font=small}
\pagestyle{fancy}
\renewcommand{\headrulewidth}{0pt}
\lhead{}\chead{}\rhead{}
\lfoot{}\cfoot{\thepage}\rfoot{}
\usepackage{sectsty}
\usepackage[nottoc,notlof,notlot]{tocbibind}
\usepackage[titles,subfigure]{tocloft}
\renewcommand{\cftsecfont}{\rmfamily\mdseries\upshape}
\renewcommand{\cftsecpagefont}{\rmfamily\mdseries\upshape}

\let\oldsection\section% Store \section
\renewcommand{\section}{% Update \section
	\renewcommand{\theequation}{\thesection.\arabic{equation}}% Update equation number
	\oldsection}% Regular \section
\let\oldsubsection\subsection% Store \subsection
\renewcommand{\subsection}{% Update \subsection
	\renewcommand{\theequation}{\thesubsection.\arabic{equation}}% Update equation number
	\oldsubsection}% Regular \subsection

\newcommand{\abs}[1]{\left\lvert#1\right\rvert}
\newcommand{\norm}[1]{\left\lVert#1\right\rVert}

\newcommand{\g}{\text{g}}
\newcommand{\m}{\text{m}}
\newcommand{\cm}{\text{cm}}
\newcommand{\mm}{\text{mm}}
\newcommand{\s}{\text{s}}
\newcommand{\N}{\text{N}}
\newcommand{\Hz}{\text{Hz}}

\newcommand{\virgolette}[1]{``\text{#1}"}
\newcommand{\tildetext}{\raise.17ex\hbox{$\scriptstyle\mathtt{\sim}$}}


\renewcommand{\arraystretch}{1.2}

\addto\captionsenglish{\renewcommand{\figurename}{Fig.}}
\addto\captionsenglish{\renewcommand{\tablename}{Tab.}}

\DeclareCaptionLabelFormat{andtable}{#1~#2  \&  \tablename~\thetable}

%opening
\title{%
    \Huge Misura delle lunghezze d'onda con lo spettrometro a reticolo \\
    \Large Laboratorio di Ottica, Elettronica e Fisica Moderna \\ C.d.L. in Fisica, a.a. 2023-2024 \\ Università degli Studi di Milano}
\author{\LARGE Lucrezia Bioni, Leonardo Cerasi, Giulia Federica Bianca Coppi \\ Matricole: 13655A, 11410A, 11823A}
\date{16 novembre 2023}


\begin{document}

    \maketitle

    \section{Introduzione}

    \subsection{Scopo}
    Lo scopo dell'esperienza è la misura delle lunghezze d'onda di alcune righe dello spettro di una sorgente di mercurio attraverso un reticolo in precedenza tarato con il doppietto del sodio.

    \subsection{Metodo}
    Un reticolo è un dispositivo che si presta per la misura delle lunghezze d'onda della luce ad esso incidente. Infatti, per via dei fenomeni di interferenza e diffrazione, produce un pattern caratterizzato da un'immagine centrale non dispersa e una sequenza, simmetrica rispetto al centro, di spettri. Questi sono composti da righe colorate, ciascuna corrispondente a un massimo delle varie lunghezze d'onda costituenti la luce incidente. \\
    La posizione dei vari massimi della figura di interferenza dipende dal valore della lunghezza d'onda da cui sono generati. Dunque, dopo aver determinato il passo $d$ del reticolo in uso e dopo aver misurato, ponendo il reticolo sulla piattaforma di uno spettrometro, la posizione angolare di un massimo di ordine $k=\pm 1, \pm 2,\ldots$ rispetto al massimo centrale ($\Delta \theta$), è possibile determinare la lunghezza d'onda $\lambda$ della componente del fascio incidente responsabile di quella specifica riga di spettro:
    \begin{equation}
        \label{lambda}
        \lambda = \frac{d \, \, \sin{\Delta\theta}}{k}
    \end{equation}
    Il passo $d$ del reticolo si ottiene dalle misure delle posizioni angolari dei massimi di interferenza del doppietto del sodio, le cui lunghezze d'onda si assumono note:
    \begin{equation}
        \label{lambda-sodio}
        \lambda_1 = 5.890 \cdot 10^{-7} \text{m}
    \end{equation}


\end{document}