\documentclass[]{article}
\usepackage[utf8]{inputenc}
\usepackage[english]{babel}

\usepackage[]{csvsimple}

\usepackage{ragged2e}
\usepackage[left=25mm, right=25mm, top=15mm]{geometry}
\geometry{a4paper}
\usepackage{graphicx}
\usepackage{booktabs}
\usepackage{paralist}
\usepackage{subfig} 
\usepackage{fancyhdr}
\usepackage{amsmath}
\usepackage{amssymb}
\usepackage{amsfonts}
\usepackage{amsthm}
\usepackage{mathtools}
\usepackage{enumitem}
\usepackage{titlesec}
\usepackage{braket}
\usepackage{gensymb}
\usepackage{url}
\usepackage{hyperref}
\usepackage{csquotes}
\usepackage{multicol}
\usepackage{graphicx}
\usepackage{wrapfig}
\usepackage{babel}
\usepackage{caption}
\captionsetup{font=small}
\pagestyle{fancy}
\renewcommand{\headrulewidth}{0pt}
\lhead{}\chead{}\rhead{}
\lfoot{}\cfoot{\thepage}\rfoot{}
\usepackage{sectsty}
\usepackage[nottoc,notlof,notlot]{tocbibind}
\usepackage[titles,subfigure]{tocloft}
\renewcommand{\cftsecfont}{\rmfamily\mdseries\upshape}
\renewcommand{\cftsecpagefont}{\rmfamily\mdseries\upshape}

\let\oldsection\section% Store \section
\renewcommand{\section}{% Update \section
	\renewcommand{\theequation}{\thesection.\arabic{equation}}% Update equation number
	\oldsection}% Regular \section
\let\oldsubsection\subsection% Store \subsection
\renewcommand{\subsection}{% Update \subsection
	\renewcommand{\theequation}{\thesubsection.\arabic{equation}}% Update equation number
	\oldsubsection}% Regular \subsection

\newcommand{\abs}[1]{\left\lvert#1\right\rvert}
\newcommand{\norm}[1]{\left\lVert#1\right\rVert}

\newcommand{\g}{\text{g}}
\newcommand{\m}{\text{m}}
\newcommand{\cm}{\text{cm}}
\newcommand{\mm}{\text{mm}}
\newcommand{\s}{\text{s}}
\newcommand{\N}{\text{N}}
\newcommand{\Hz}{\text{Hz}}

\newcommand{\virgolette}[1]{``\text{#1}"}
\newcommand{\tildetext}{\raise.17ex\hbox{$\scriptstyle\mathtt{\sim}$}}


\renewcommand{\arraystretch}{1.2}

\addto\captionsenglish{\renewcommand{\figurename}{Fig.}}
\addto\captionsenglish{\renewcommand{\tablename}{Tab.}}

\DeclareCaptionLabelFormat{andtable}{#1~#2  \&  \tablename~\thetable}

%opening
\title{%
    \Huge Misura del rapporto carica su massa di elettroni non relativistici \\
    \Large C.d.L. in Fisica, a.a. 2022-2023 \\ Università degli Studi di Milano}
\author{\LARGE Lucrezia Bioni, Leonardo Cerasi, Giulia Federica Bianca Coppi}
\date{}

\begin{document}

    \maketitle

    \section{Introduzione}

    L'elettrone è una particella carica e massiva: in questa esperienza ci si propone di misurare il suo rapporto carica-massa $ \frac{e}{m} $ in condizioni non relativistiche. \\
    Le misurazioni vengono effettuate in tre casi distinti: perpendicolarmente, parallelamente e antiparallelamente al campo magnetico terrestre.


    \subsection{Metodo}

    In un'ampolla contenente gas idrogeno a bassa pressione (circa $ 10^{-2}\, torr $), è posta una resistenza che, essendo percorsa da corrente elettrica alternata, scalda un catodo che produce elettroni per effetto termoelettrico. 
    Questi, accelerati dalla differenza di potenziale $\Delta V$ presente tra il catodo e l'anodo, vengono fatti collimare in un unico fascio, la cui traiettoria viene deviata dalla forza di Lorentz - ortogonale al vettore velocità degli elettroni - generata dal campo magnetico $ B_z $ prodotto dalle bobine di Helmholz poste ai lati dell'ampolla: il cammino percorso dal fascio assume una forma circolare grazie alla regolazione dell'intensità di $B_z$. \\

    Gli elettroni, sul loro percorso, collidono contro le molecole di idrogeno, emettendo fotoni ad una lunghezza d'onda di circa 4500 Å.
    Il raggio della circonferenza visibile risulta fondamentale per la determinazione della grandezza interessata come mostrato dalla seguente equazione:

    \begin{equation}
        \label{e_m}
        \frac{e}{m} = \frac{2 \Delta V}{(B_z R)^2}
    \end{equation}

    dove $ e $ ed $ m $ rappresentano rispettivamente la carica elettrica e la massa dell'elettrone, $\Delta V$ rapprensenta la differenza di potenziale, $B_z$ rappresenta il campo elettrico generato dalle bobine di Helmholz e $ R $ rapprensenta il raggio della circonferenzza compiuta dagli elettroni. \\

    In base alla differente influenza del campo magnetico terrestre, il campo magnetico effettivo a cui è sottoposto il fascio di elettroni varia: è dunque necessario andare a stimare il contributo di quest'ultimo in modo da poter fornire un risultato più accurato al valore finale della grandezza interessata.

    \section{Misure}

    La determinazione delle grandezze utili a fornire un valore di $e/m$ richiede la determinazione preliminare del raggio medio delle bobine di Helmholz: tale misurazione viene compiuta attraverso l'utilizzo di un calibro di risoluzione 0.02 mm. Le bobine di Helmholz hanno la caratteristica di essere poste ad una distanza equivalente al loro raggio medio, dunque si procede misurando la loro distanza per eccesso e per difetto - 5 misurazioni per entrambe le distanze - e come valore finale si fornisce la media delle due medie ottenute dai due set di dati. Come incertezza, invece, viene fornita la somma in quadratura delle devaizioni standard dei due set si dati:

    \begin{equation}
        \label{misura_Rb}
        R_b = (15.63 \pm 0.07) \, \text{cm}
    \end{equation} 

    Tutte le misure effettuate sono riportate nella \ref{Raggio_bobine}

    Si è poi misurata la distanza $A$ tra la seconda lente dell'apparato e lo specchio rotante. La misura è stata effettuata
    attraverso un calibro di risoluzione $ 0,01 \, \text{m} $, che è stata attribuita come incertezza di $a$.
    
    \begin{equation}
        \label{equation for a}
        a = (0.474 \pm 0.001) \, \text{m}
    \end{equation} \\
    Dopo aver avviato lo specchio rotante in senso orario a una frequenza di rotazione $ \nu_0 $ nell'intervallo $[10,20] \, \text{Hz}$,
    si è misurata, mediante micrometro di risoluzione $ 0.00001 \text{m} $ , la posizione dello spot luminoso $\delta_0$ visibile attraverso
    un microscopio. Si è poi portato lo specchio in un intorno della frequenza massima di rotazione $ \nu $ e si è misurata,
    sempre mediante micrometro, la nuova posizione dello spot luminoso $\delta$. Tale set di misure è stato ripetuto per $30$ volte.
    I dati rilevati sono riportati nella tabella $ riferimento-alla-tabella $. \\ 

    Si sono prese le medesime misure di posizione, a frequenza minima e a frequenza massima, facendo ruotare
    lo specchio in senso antiorario. I dati rilevati sono riportati nella tabella $ riferimento-alla-tabella $. \\ 

    Si sono poi effettuate misure di posizione dello spot luminoso portando lo specchio dalla frequenza massima di rotazione in senso
    orario $\nu_0$ alla frequenza massima di rotazione in senso antiorario $\nu$.
    I dati rilevati sono riportati nella tabella $ riferimento-alla-tabella $. \\

    Infine, si sono rilevate misure di posizione dello spot luminoso con lo specchio a frequenza minima di rotazione ( $[10,20] \, \text{Hz}$ )
    e a frequenze intermedie, sia in senso orario (i dati sono riportati nella tabella $ riferimento-alla-tabella $) sia 
    in senso antiorario (i dati sono riportati nella tabella $ riferimento-alla-tabella $). \\

    \section {Analisi dati}

    \subsection {Stima degli errori}

    \section{Appendice}

    \begin{table}
        \centering

    \begin{tabular}{||c|c||}
        \hline
        $R_b \, per \, difetto\, [\text{cm}] $ & $R_b \, per \, eccesso\, [\text{cm}] $\\
        \hline\hline

        $ 13.45 $ & $ 17.77 $ \\\hline
        $ 13.51 $ & $ 17.80 $ \\\hline
        $ 13.48 $ & $ 17.69 $ \\\hline
        $ 13.76 $ & $ 17.74 $ \\\hline
        $ 13.43 $ & $ 17.70 $ \\\hline
    
    \end{tabular}
    \caption{Misure del raggio medio delle bobine di Helmholz per difetto e per eccesso.}
    \label{Raggio_bobine}
\end{table}

\begin{table}
    \centering

    \begin{tabular} {||c|c||c|c||}
        \hline
        $ media \, per \, difetto\, [text{cm}] $ & $\sigma_dif [text{cm}] $ & $ media \, per \, eccesso\, [text{cm}] $ & $\sigma_ecc [text{cm}] $

        $ 13.52 $ & $ 0.14 $ & $ 17.74 $ & $ 0.05 $ \\\hline

    \end{tabular}
    \caption{Medie per difetto e per eccesso con relative deviazioni standard}
    \label{media_devst_Rb}

\end{table}


    \begin{table}
        \centering

    \begin{tabular}{||c|c|c|c|c|c|c||}
        \hline
        $\Delta V\, [\text{V}] $ & $d\, [\text{m}] $ & $R\, [\text{m}] $ & $I\, [\text{A}] $ & $Termine \, correttivo $ & $B(0)\, [\text{A/m}] $ & $B(R)\, [\text{A/m}] $\\
        \hline\hline


 
        $334,4$ & $0,09542$ & $0,04771 $ & $1,667$ & $0,99567 $ & $0,00124662 $ & $0,001241223$ \\\hline
        $275,0$ & $0,09542$ & $0,04771 $ & $1,545$ & $0,99567 $ & $0,001155386$ & $0,001150383$ \\\hline
        $297,8$ & $0,09814$ & $0,04907 $ & $1,497$ & $0,995265$ & $0,001119491$ & $0,00111419 $ \\\hline
        $321,5$ & $0,09814$ & $0,04907 $ & $1,597$ & $0,995265$ & $0,001194273$ & $0,001188618$ \\\hline
        $300,6$ & $0,09542$ & $0,04771 $ & $1,518$ & $0,99567 $ & $0,001135195$ & $0,001130279$ \\\hline
        $258,3$ & $0,09814$ & $0,04907 $ & $1,393$ & $0,995265$ & $0,001041717$ & $0,001036785$ \\\hline
        $298  $ & $0,09814$ & $0,04907 $ & $1,545$ & $0,995265$ & $0,001155386$ & $0,001149915$ \\\hline
        $330,5$ & $0,09814$ & $0,04907 $ & $1,628$ & $0,995265$ & $0,001217455$ & $0,001211691$ \\\hline
        $386,6$ & $0,09814$ & $0,04907 $ & $1,739$ & $0,995265$ & $0,001300464$ & $0,001294306$ \\\hline
        $353,6$ & $0,09814$ & $0,04907 $ & $1,689$ & $0,995265$ & $0,001263073$ & $0,001257092$ \\\hline
        $310,8$ & $0,093  $ & $0,0465  $ & $1,639$ & $0,995265$ & $0,001225681$ & $0,001219878$ \\\hline
        $340,4$ & $0,1101 $ & $0,05505 $ & $1,503$ & $0,995265$ & $0,001123978$ & $0,001118656$ \\\hline
        $365,7$ & $0,08635$ & $0,043175$ & $1,859$ & $0,995265$ & $0,001390202$ & $0,00138362 $ \\\hline
        $334,5$ & $0,09814$ & $0,0465  $ & $1,692$ & $0,995265$ & $0,001265316$ & $0,001259325$ \\\hline
        $302,2$ & $0,09814$ & $0,04907 $ & $1,575$ & $0,99486 $ & $0,001177821$ & $0,001171767$ \\\hline
        $326,5$ & $0,09814$ & $0,04907 $ & $1,637$ & $0,995265$ & $0,001224186$ & $0,001218389$ \\\hline
        $344  $ & $0,09814$ & $0,04907 $ & $1,721$ & $0,995265$ & $0,001287003$ & $0,001280909$ \\\hline
        $358,4$ & $0,09814$ & $0,04907 $ & $1,779$ & $0,995265$ & $0,001330377$ & $0,001324077$ \\\hline
        $342,7$ & $0,09814$ & $0,04907 $ & $1,676$ & $0,995265$ & $0,001253351$ & $0,001247416$ \\\hline
        $314  $ & $0,09814$ & $0,04907 $ & $1,617$ & $0,995265$ & $0,001209229$ & $0,001203504$ \\\hline
    
    \end{tabular}
    \caption{Campo Magnetico terrestre ortogonale al campo magnetico generato dalle bobine di Helmholz. Si riportano i valori della differenza di potenziale $\Delta V$, del diametro $ d $ della circonferenza percorsa dagli elettroni e del conseguente raggio $ R $, dell'intensità di corrente $ I $ e del termine correttivo utilizzato per correggere il campo magnetico $B(0)$ - colonna 5 - all'effettivo valore B(R) - colonna 6 -.}
    \label{CM_ortogonale}
\end{table}



\end{document}