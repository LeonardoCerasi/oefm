\documentclass[]{article}
\usepackage[utf8]{inputenc}
\usepackage[english]{babel}

\usepackage[]{csvsimple}
\usepackage{float}

\usepackage{ragged2e}
\usepackage[left=25mm, right=25mm, top=15mm]{geometry}
\geometry{a4paper}
\usepackage{graphicx}
\usepackage{booktabs}
\usepackage{paralist}
\usepackage{subfig} 
\usepackage{fancyhdr}
\usepackage{amsmath}
\usepackage{amssymb}
\usepackage{amsfonts}
\usepackage{amsthm}
\usepackage{mathtools}
\usepackage{enumitem}
\usepackage{titlesec}
\usepackage{braket}
\usepackage{gensymb}
\usepackage{url}
\usepackage{hyperref}
\usepackage{csquotes}
\usepackage{multicol}
\usepackage{graphicx}
\usepackage{wrapfig}
\usepackage{babel}
\usepackage{caption}
\captionsetup{font=small}
\pagestyle{fancy}
\renewcommand{\headrulewidth}{0pt}
\lhead{}\chead{}\rhead{}
\lfoot{}\cfoot{\thepage}\rfoot{}
\usepackage{sectsty}
\usepackage[nottoc,notlof,notlot]{tocbibind}
\usepackage[titles,subfigure]{tocloft}
\renewcommand{\cftsecfont}{\rmfamily\mdseries\upshape}
\renewcommand{\cftsecpagefont}{\rmfamily\mdseries\upshape}

\let\oldsection\section% Store \section
\renewcommand{\section}{% Update \section
	\renewcommand{\theequation}{\thesection.\arabic{equation}}% Update equation number
	\oldsection}% Regular \section
\let\oldsubsection\subsection% Store \subsection
\renewcommand{\subsection}{% Update \subsection
	\renewcommand{\theequation}{\thesubsection.\arabic{equation}}% Update equation number
	\oldsubsection}% Regular \subsection

\newcommand{\abs}[1]{\left\lvert#1\right\rvert}
\newcommand{\norm}[1]{\left\lVert#1\right\rVert}

\newcommand{\g}{\text{g}}
\newcommand{\m}{\text{m}}
\newcommand{\cm}{\text{cm}}
\newcommand{\mm}{\text{mm}}
\newcommand{\s}{\text{s}}
\newcommand{\N}{\text{N}}
\newcommand{\Hz}{\text{Hz}}

\newcommand{\virgolette}[1]{``\text{#1}"}
\newcommand{\tildetext}{\raise.17ex\hbox{$\scriptstyle\mathtt{\sim}$}}


\renewcommand{\arraystretch}{1.2}

\addto\captionsenglish{\renewcommand{\figurename}{Fig.}}
\addto\captionsenglish{\renewcommand{\tablename}{Tab.}}

\DeclareCaptionLabelFormat{andtable}{#1~#2  \&  \tablename~\thetable}


\title{%
    \Huge Interferometro di Michelson \\
    \Large Laboratorio di Ottica, Elettronica e Fisica Moderna \\ C.d.L. in Fisica, a.a. 2023-2024 \\ Università degli Studi di Milano}
\author{\LARGE Lucrezia Bioni, Leonardo Cerasi, Giulia Federica Bianca Coppi \\ Matricole: 13655A, 11410A, 11823A}
\date{9 novembre 2023}

\begin{document}

\maketitle

\section{Introduzione}

\subsection{Scopo}

In questa esperienza ci si propone di misurare - mediante l'utilizzo dell'interferometro di Michelson - le seguenti quattro quantità: la lunghezza d'onda di un fascio di luce monocromatica, l'indice di rifrazione dell'aria a pressione atmosferica, la lunghezza dei pacchetti d'onda di una sorgente non monocromatica e la separazione tra le due lunghezze d'onda del doppietto del sodio.

\subsection{Metodo}

Per la misurazione delle quattro grandezze interessate, si utilizza l'apparato sviluppato da Michelson riportato in figura $riferimento$. L'interferometro è costituito da quattro lastre di vetro ($S_1, \, S_2, \, S_3, \, L_c$): $S_1$ è una lastra semiriflettente - rivolta verso $S_2$ -a facce piane e parallele, $S_2$ e $S_3$ sono completamente riflettenti sulla faccia rivolta verso $S_1$, $L_c$ è una lastra trasparente il cui scopo è quello di rendere uguali i cammini ottici compiuti dai raggi lungo i due bracci dello strumento. \\ Essendosi assicurati che $S_2$ e $S_3$ siano perpendicolari e che formino un angolo di $45°$ con $S_1$, il raggio luminoso inciderà su $S_1$ sdoppiandosi: il primo verrà riflesso da $S_2$ e dalla faccia riflettente di $S_1$, per poi proseguire verso lo schermo, il secondo - riflesso da $S_1$ - verrà riflesso da $S_3$ ed inciderà sullo schermo dove formerà delle figure di interferenza con il primo raggio - douvuta alla coerenza dei due fasci luminosi- .

\subsubsection{Misura della lunghezza d'onda di un fascio di luca monocromatica}

Si vuole misurare la lunghezza d'ond di un fascio di luce laser: agendo sulla variazione di cammino ottico dei due fasci - spostando lo specchio $S_3$ - si conta il numero di frange chiare (o scure) passanti per un punto prefissato dello schermo. La misura della lunghezza d'onda è pertanto data dalla formula

\begin{equation}
    \label{lambda_laser}
    \lambda = \frac{2 n_a \Delta x}{N_1}
\end{equation}

dove $\lambda$ è la lunghezza d'onda incognita, $n_a$ è l'indice di rifrazione dell'aria, $\Delta x$ è lo spostamento dello specchio $S_3$ e $N_1$ è il numero di frange chiare (o scure) contate.

\subsubsection{Misura dell'indice di rifrazione dell'aria}

Tra gli specchi $S_1$ e $S_2$ viene inserita una cameretta contenente una pompa per la creazione del vuoto. Il cammino ottico percorso dal fascio luminoso nel vuoto cambia - poichè questo è legato all'indice di rifrazione del mezzo che attraversa come mostrato dall'equazione \ref{lambda_laser} - e quindi, facendo rientrare lentamente l'aria nella cameretta e contando le frange di interferenza passanti per un dato punto sullo schermo, si riuscirà a fornire una stima del valore dell'indice di rifrazione dell'aria $n_a$ seocndo la seguente equazione:

\begin{equation}
    \label{n_a}
    2(n_a - 1) = N_2 \lambda
\end{equation}

dove $n_a$ è l'indice di rifrazione dell'aria, $N_2$ è il numero di frange contate su un punto dello schermo e $\lambda$ è la lunghezza d'onda del fascio emesso dalla sorgente monocromatica.

\subsubsection{Misura della lunghezza dei pacchetti d'onda di una sorgente non monocromatica}

Il fascio di luce prodotto da una sorgente non monocromatica è costituita da impulsi di lunghezza limitata. L'inferferenza dei fasci luminosi riflessi dagli specchi $S_2$ e $S_3$ si manifesta quando la distanza tra le due sorgenti immagine è inferiore alla lunghezza del pacchetto: quando viene superata tale lunghezza, si osserva sullo schermo una figura unifermemente illuminata e quindi si misura la distanza tra due zone di uniforme illuminazione - mediante la misura dello spostamento di $S_3$ - per quantificare tale grandezza.

\subsubsection{Misura della separazione tra le due lunghezze d'onda del doppietto del sodio}

Si utilizza ora una sorgente luinosa al sodio per misurare le due lunghezze d'onda che emette e la loro cnseguente separazione: quando le frange di interferenza delle due lunghezze d'onda si vanno a sovrapporre, sullo schermo si vede una figura di interferenza con frange molto nette - in particolare quando la differenza di cammino ottico trta i fasci prpvenieni da $S_2$ ed $S_3$ è nulla -. 
Si misura quindi lo spostamento dello specchio $S_3$ e di ricava:

\begin{equation}
    \label{Delta_lambda}
    \lambda _2 - \lambda _1 = \frac {m \lambda ^2 }{2 \Delta x}
\end{equation}

dove $\lambda _1$ e $\lambda _2$ sono le due lunghezze d'onda del doppietto del sodio, $m$ è il numero di alternanze tra le condizioni di interferenza netta, $\lambda$ è la media delle due lunghezze d'onda e $\Delta x$ è lo spostamento dello spcchio $S_3$.

\section{Misure}

\subsection {Misura della lunghezza d'onda di un fascio di luce monocromatica}

La misura della lunghezza d'onda del fascio laser viene effettuata prendendo 5 misure dello spostamento dello specchio mobile e contando le frange passanti per un punto fissato dello schermo, le misure sono riportate nella seguente Tabella.

\begin{table}[H]
    \centering

    \begin{tabular}{||c|c|c||}
        \hline
        $N_1 $ & $x_1 \, \text{[mm]}$ & $x_2\, \text{[mm]}$ \\
        \hline\hline

        $195$ & $10.00$ & $10.30$ \\\hline
        $194$ & $10.00$ & $10.30$ \\\hline
        $150$ & $10.00$ & $10.23$ \\\hline
        $150$ & $10.00$ & $10.23$ \\\hline
        $180$ & $10.00$ & $10.28$ \\\hline
    
    \end{tabular}
    \caption{Misure di $N_1$, $x_1$ e $x_2$ effetuate per valutare la lunghezza d'onda della sorgente laser}
    \label{lambda}    
\end{table}

Al conteggio $N_1$ viene fornito un errore di $ \pm 5$, alle misure di $x_1$ e $x_2$ viene fornita l'incertezza strumentale pari a $0.01 mm$ 


\subsection{Misura dell'indice di rifrazione dell'aria}

A camera usata per creare il vuoto risulta essere lunga $D=0.05 m$ - valore fornito - e per fornire un valore ad $n_a$ si contano 5 volte il numero di frange di interferenza passanti epr un punto dello schermo. I valori ottenuti sono tutti uguali e pertanto si fornisce il valore $N_2 = 42$ a cui non viene attribuito alcun errore.

\subsection{Misura della lunghezza dei pacchetti d'onda di una sorgente non monocromatica}

Vengono fatte 6 misure dello spostamento dello specchio per valutare la lunghezza del treno di impulsi come descritto nel paragrafo 1.2.3. I risultati sono riportati in tabella.

\begin{table}[H]
    \centering

    \begin{tabular}{||c|c||}
        \hline
        $x_1 \, \text{[mm]}$ & $x_2\, \text{[mm]}$ \\
        \hline\hline

        $15.58$ & $15.54$ \\\hline
        $15.58$ & $15.54$ \\\hline
        $15.57$ & $15.54$ \\\hline
        $15.57$ & $15.54$ \\\hline
        $15.57$ & $15.54$ \\\hline
        $15.57$ & $15.54$ \\\hline
    
    \end{tabular}
    \caption{Misure della posizione iniziale e finale dello specchio $S_3$}
    \label{L}    
\end{table}

A queste misure viene sempre fornita l'incertezza strumentale pari a $0.01mm$.

\subsection{Misura della separazione tra le due lunghezze d'onda del doppietto del sodio}

Per valutare la distanza delle due lunghezze d'onda emesse dal sodio vegono prese 8 misure dello spostamento dello specchio $S_3$, fornendo anche il valore $m$ di numero di alternanze di interferenze nette viste sullo schermo durante lo spostamento dello specchio mobile. Le misure vengono riportate in tabella.

\begin{table}[H]
    \centering

    \begin{tabular}{||c|c|c||}
        \hline
        $m $ & $x_1 \, \text{[mm]}$ & $x_2\, \text{[mm]}$ \\
        \hline\hline

        $1$ & $16.24$ & $17.73$ \\\hline
        $1$ & $17.73$ & $19.11$ \\\hline
        $1$ & $19.11$ & $20.66$ \\\hline
        $1$ & $20.66$ & $22.07$ \\\hline
        $1$ & $22.07$ & $23.58$ \\\hline
        $1$ & $23.58$ & $24.98$ \\\hline
        $1$ & $17.72$ & $19.15$ \\\hline
        $2$ & $19.15$ & $22.17$ \\\hline
    
    \end{tabular}
    \caption{Misure di $m$, $x_1$ e $x_2$ effettuate per valutare $\Delta \lambda$ del doppietto di $Na$ }
    \label{sodio}    
\end{table}

Alle misure di $m$ non viene fornita alcuna incertezza, a quelle di $x_1$ e $x_2$ viene fornita l'incertezza strumentale di $0.01mm$.


\end{document}