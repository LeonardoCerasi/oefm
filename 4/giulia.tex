\documentclass[]{article}
\usepackage[utf8]{inputenc}
\usepackage[english]{babel}

\usepackage[]{csvsimple}
\usepackage{float}

\usepackage{ragged2e}
\usepackage[left=25mm, right=25mm, top=15mm]{geometry}
\geometry{a4paper}
\usepackage{graphicx}
\usepackage{booktabs}
\usepackage{paralist}
\usepackage{subfig} 
\usepackage{fancyhdr}
\usepackage{amsmath}
\usepackage{amssymb}
\usepackage{amsfonts}
\usepackage{amsthm}
\usepackage{mathtools}
\usepackage{enumitem}
\usepackage{titlesec}
\usepackage{braket}
\usepackage{gensymb}
\usepackage{url}
\usepackage{hyperref}
\usepackage{csquotes}
\usepackage{multicol}
\usepackage{graphicx}
\usepackage{wrapfig}
\usepackage{babel}
\usepackage{caption}
\captionsetup{font=small}
\pagestyle{fancy}
\renewcommand{\headrulewidth}{0pt}
\lhead{}\chead{}\rhead{}
\lfoot{}\cfoot{\thepage}\rfoot{}
\usepackage{sectsty}
\usepackage[nottoc,notlof,notlot]{tocbibind}
\usepackage[titles,subfigure]{tocloft}
\renewcommand{\cftsecfont}{\rmfamily\mdseries\upshape}
\renewcommand{\cftsecpagefont}{\rmfamily\mdseries\upshape}

\let\oldsection\section% Store \section
\renewcommand{\section}{% Update \section
	\renewcommand{\theequation}{\thesection.\arabic{equation}}% Update equation number
	\oldsection}% Regular \section
\let\oldsubsection\subsection% Store \subsection
\renewcommand{\subsection}{% Update \subsection
	\renewcommand{\theequation}{\thesubsection.\arabic{equation}}% Update equation number
	\oldsubsection}% Regular \subsection

\newcommand{\abs}[1]{\left\lvert#1\right\rvert}
\newcommand{\norm}[1]{\left\lVert#1\right\rVert}

\newcommand{\g}{\text{g}}
\newcommand{\m}{\text{m}}
\newcommand{\cm}{\text{cm}}
\newcommand{\mm}{\text{mm}}
\newcommand{\s}{\text{s}}
\newcommand{\N}{\text{N}}
\newcommand{\Hz}{\text{Hz}}

\newcommand{\virgolette}[1]{``\text{#1}"}
\newcommand{\tildetext}{\raise.17ex\hbox{$\scriptstyle\mathtt{\sim}$}}


\renewcommand{\arraystretch}{1.2}

\addto\captionsenglish{\renewcommand{\figurename}{Fig.}}
\addto\captionsenglish{\renewcommand{\tablename}{Tab.}}

\DeclareCaptionLabelFormat{andtable}{#1~#2  \&  \tablename~\thetable}


\title{%
    \Huge Misura del rapporto carica/massa di un elettrone non relativistico \\
    \Large Laboratorio di Ottica, Elettronica e Fisica Moderna \\ C.d.L. in Fisica, a.a. 2023-2024 \\ Università degli Studi di Milano}
\author{\LARGE Lucrezia Bioni, Leonardo Cerasi, Giulia Federica Bianca Coppi \\ Matricole: 13655A, 11410A, 11823A}
\date{2 novembre 2023}

\begin{document}

\maketitle

\section{Introduzione}

\subsection{Scopo}

L'interferometro di Michelson è uno strumento che permette di misurare le seguenti quantità: la lunghezza d'onda di un fascio di luce monocromatica, l'indice di rifrazione dell'aria a pressione atmosferica, la lunghezza dei pacchetti d'onda di una sorgente non monocromatica e la separazione tra le due lunghezze d'onda del doppietto del sodio.

\subsection{Metodo}

Per la msurazione delle quattro grandezze interessate, si utilizza l'apparato sviluppato da Michelson riportato in figura. L'interferometro è costituito da quattri lastre di vetro ($S_1, \, S_2, \, S_3, \, L_c$): $S_1$ è una lastra semiriflettente - rivolta verso $S_2$ -a facce piane e parallele, $S_2$ e $S_3$ sono completamente riflettenti sulla faccia rivolta verso $S_1$, $L_c$ è una lastra trasparente il cui scopo è quello di rendere uguali i cammiini ottici compiuti dai raggi lungo i due braci dello strumento. 
Essendosi assicurati che $S_2$ e $S_3$ siano perpendicolari e che formino un angolo di $45°$ con $S_1$, il raggio luminoso inciderà su $S_1$ sdoppiandosi: il primo verrà riflesso da $S_2$ e dalla faccia riflettente di $S_1$, per poi proseguire verso lo schermo, il secondo - riflesso da $S_1$ - verrà riflesso da $S_3$ ed inciderà sullo schermo dove formerà delle figure di interferenza con il primo raggio - douvuta alla coerenza dei due fasci luminosi- .\\

\section{Misure}

\subsection {Misura della lunghezza d'onda di un fascio di luce monocromatica}

Si vuole misurare la lunghezza d'ond di un fascio di luce laser: agendo sulla variazione di cammino ottico dei due fasci - spostando lo specchio $S_3$ - si conta il numero di frange chiare (o scure) passanti per un punto prefissato dello schermo. la misura della lunghezza d'onda è pertanto data dalla formula

\begin{equation}
    \label{lambda_laser}
    \lambda = \frac{2 n_a \Delta x}{N_1}
\end{equation}

dove $\lambda$ è la lunghezza d'onda incognita, $n_a$ è l'indice di rifrazione dell'aria, $\Delta x$ è lo spostamento dello specchio $S_3$ e $N_1$ è il numero di frange chiare (o scure) contate.



\section{Appendice}


\end{document}